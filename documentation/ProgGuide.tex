\documentclass[a4paper,12pt]{report}
\usepackage{ngerman}

%neue deutsche Rechtschreibung
%\usepackage[english,ngerman]{babel}

\usepackage[latin1]{inputenc} %f�r Linux
%\usepackage[ansinew]{inputenc} %fuer Windows

\usepackage{mycore}

\begin{document}
%-------   Vorspann
\title{MyCoRe Internal Design Guilde}
\author{
Johannes B�hler
            }
\maketitle

\setcounter{secnumdepth}{10} 


\chapter*{Vorwort} 

\tableofcontents
\listoffigures
\listoftables

\chapter{ Einf"uhrung und Grundprinzipien} 
\section{XML/XSLT}
\section{XPath/Queries}
\section{Sessionmodell}
\section{Das Vererbungsmodell}
\section{Das API-Konzept allgemein}
\section{Klassen und Verantwortlichkeiten }
\section{Allgemeine Klassen / Exception-Modell / MCRCache}
\section{Das Metadatenmodell}
\subsection{MCRClassification}
\subsection{MCRObject}
\subsection{MCRDerivate}
\subsection{MCRLinkManager}
\subsection{Erweiterungsm"oglichkeiten}
Seitenumbruch erzwingen?
\section{Das IFS Modell (ohne Stores)}
\section{Das User- und ACL-Modell}
\section{Der Backend-Store}
\subsection{CM7}
\subsection{CM8.2 (inklusive Datenspeichermodell und Query Umsetztung)}
\subsection{XMLDB (inklusive Datenspeichermodell und Query Umsetztung)}
\subsection{DB2 / MySQL / Oracle}
\subsection{Filesystem-Store}
\subsection{VideoStores}
\subsection{RemoteStore im Detail}
\subsection{Weitere Ausbaum�glichkeiten}
\section{Die Frontend Komponenten}
\subsection{Das Commandline-Tool und seine Erweiterungen}
\subsection{Zusammenspiel der Servlets und Funktion der einzelnen}
\subsection{Login-Servlet und MCRSession}
\subsection{IFS-Servlet}
\subsection{Query-Servlet}
\subsection{Innere Struktur des Editor-Servlets}
\subsection{Innere Struktur des User-Servlets}
\section{Funktionsprinzipen der Services}
\subsection{OAI}
\subsection{NBN}
\subsection{Weitere}
\chapter{Konfiguration im Detail}
\section{Mycore.properties}
\section{Suchmasken}
\section{Stylesheets}
% Glossar
%
% UserGuide - Glossar
%
\chapter*{Glossar}
{\bf NBN} \\[1.5ex]
Was ist eigentlich NBN??? \\[2ex]
{\bf OAI} \\[1.5ex]
Was ist eigentlich OAI??? \\[2ex]
{\bf XML} \\[1.5ex]
Was ist eigentlich XML??? \\[2ex]
{\bf XSLT} \\[1.5ex]
Was ist eigentlich XSLT??? \\[2ex]
%
%

\end{document}




