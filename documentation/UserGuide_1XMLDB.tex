%
%
\section{Hinweise zur Installation einer freier und komerzieller Datenbanken, XML:DB's und Web-Komponenten}
%
%
\subsection{Installation der Komponenten unter Linux}
Die folgende Beschreibung erl�utert die Arbeit unter SuSE Linux, getestet wurde unter der Version 8.1. Sollten f�r RedHat Abweichungen auftreten, so werder diese gesondert vermerkt.
%
%
\subsubsection{Vorbereitung}
%
%
Als erster Schritt sind der Java-Compiler J2SE 1.3.1 oder h�her und ANT zu installieren. Beide befinden sich unter in der SuSE in der Distribution. 
\begin{itemize}
\item {\bf java2-jre-1.3.1-521} oder h�her
\item {\bf java2-1.3.1-521} oder h�her
\item {\bf jakarta-ant-1.5-76} oder h�her
\end{itemize}

Wenn allerdings SuSE von einem ftp Server installiert wird, so ist Java nicht standardm��ig enthalten. Dann bitte die Java Linux Version von der \url{www.blackdown.org}\footnote{\url{ftp://ftp.informatik.hu-berlin.de/pub/Java/Linux/JDK-1.4.1/i386/01/j2sdk-1.4.1-01-linux-i586-gcc3.2.bin}} Homepage runterladen und wie folgt installieren. Download nach beispielsweise nach {\it /usr/local} anschlie�end mit {\tt chmod +x j2sdk-1.4.1-01-linux-<ARCH>.bin} ausf�hrbar machen und ausf�hren {\tt ./j2sdk-1.4.1-01-linux-<ARCH>.bin}.  Damit die Pfad Variablen alle richtig gesetzt werden eine Vorlage unter {\it/etc/java} entsprechend editieren und mit {\tt setDefaultJava} als root die Links richtig setzten.
Zudem sollte man in der Datei {\it /etc/profile.local}\footnote{Falls diese noch nicht existiert einfach anlegen} noch die Variable {\it ANT\_HOME}\footnote{export ANT\_HOME=/opt/jakarta/ant} gesetzt werden. 

Sinnvollerweise sollten Sie sich einen Benutzer (z. B. {\bf mcradmin} anlegen, unter welchem Sie dann Ihre Applikationen laufen lassen.


%
%
\subsubsection{Die Installation von MySQL}
MySQL ist eine derzeit frei relationale Datenbank, welche zur schnellen Speicherung von Daten innerhalb des MyCoRe-Projektes ben�tigt wird. Sie besitzt eine JDBC Schnittstelle und ist SQL konform. Sie k�nnten MySQL auch durch eine andere verf�gbare Datenbank mit gleicher Funktionalit�t ersetzen.
\begin{enumerate}
\item Installieren Sie aus Ihrer Distribution die folgenden Pakete und danach ggf. noch vom Hersteller der Distribution per Netz angebotene Updates. Die angegebenen versionsnummern sind nur exemplarisch.
\begin{itemize}
\item {\bf mysql-3.23.52-83} oder h�her
\item {\bf mysql-shared-3.23.52-83} oder h�her
\item {\bf mysql-client-3.23.52-83} oder h�her
\item {\bf mysql-devel-3.23.52-83} oder h�her
\item {\bf mysql-bench-3.23.52-83} oder h�her
\end{itemize}
\item Die Dokumentation steht nun unter {\it /usr/share/doc/package/mysql}.
\item F�hre Sie das Kommando {\tt rcmysql } als root aus. 
\item F�hre Sie das Kommando {\tt /usr/bin/mysqladmin -u root password 'new-password'} als root aus.
\item die folgende Sequenz sorgt daf�r, dass der MyCoRe-User {\bf mcradmin} alle Rechte auf der Datenbank hat. Dabei werden bei der Ausf�hrung von Kommandos von {\bf localhost} aus keine Pssworte abgefragt. Von anderen Hosts aus muss 'ein\_password' eingegeben werden.
{\tt
\begin{verbatim}
mysql --user=root -pPASSWORD mysql
GRANT ALL PRIVILEGES ON *.* TO mcradmin@localhost WITH GRANT OPTION;
GRANT ALL PRIVILEGES ON *.* TO mcradmin@"%" IDENTIFIED BY 'ein\_passwort' WITH GRANT OPTION;
quit
\end{verbatim}
}
\item Ist das Password einmal gesetzt, m�ssen Sie zus�tzlich die Option -p verwenden.
\item Zum verifizieren, ob der Server l�uft nutzen Sie {\tt mysqladmin version} und  {\tt mysqladmin variables}.
\item jetzt k�nnen Sie die Datenbasis f�r MyCoRe mit nachstehendem Kommando anlegen.
{\tt
\begin{verbatim}
mysqladmin -u mcradmin create mycore 
\end{verbatim}
}
\item Falls weitere Benutzer noch das Recht auf Selects von allen Hosts aus haben sollen, verwenden Sie die Kommandos
{\tt
\begin{verbatim}
mysql --user=mcradmin -pPASSWORD mycore
GRANT SELECT ON mycore.* TO mycorenutzer@'%';
quit
\end{verbatim}
}
\end{enumerate}

Falls sie keine Connection auf ihren Rechnernamen (nicht localhost) aufbauen k�nnen, kann es auch mit ihrer Firewall oder TCPWrapper Einstellung zu tun haben. Bei einer Firewall sollte der Port 3306 freigegeben werden und bei einem TCPWrapper der entsprechende Dienst (mysql) in die Datei ???? geschrieben werden.
%
%
\subsubsection{Die Installation der freien XML:DB eXist}
eXist ist eine Frei verf�gbare XML:BD, welche die entsprechenden Interfaces implementiert. F�r MyCoRe wurde ein auf diesen Schnittstellen basierender Persitence-Layer implementiert. So sollte die Nutzung von eXist direkt m�glich sein.
\begin{enumerate}
\item Download der aktuellen Version von eXist von \url{http://exist-db.org/}. Achtung, da dieses produkt noch stark im Wachsen ist, sollte hier die letzte Version geholt werden. Zum Test kam Version eXist-0.9.2.
\item entpacken Sie die Distribution in ein entsprechendes Verzeichnis, z. B. unter {\it /usr/local/bin}.
\item Entfernen Sie zur Nutzung des stand-alone-Servers den Kommentar aus der Zeile {/tt uri=xmldb:exist://localhost:8081} im File {\it client.properties}.
\item Starten Sie den Server mit {\tt <installdir>/eXist-0.9.2/bin/startup.sh }.
\item unter der URL \url{http://localhost:8080/exist/index.html} sollte jetzt die eXist-Homepage erscheinen. Dies nur als Test.
\item Wenn alles okay ist, starten Sie {\tt <installdir>/eXist-0.9.2/bin/server.sh }.
\item Anschliessend k�nnen Sie auch den Client mit {\tt <installdir>/eXist-0.9.2/bin/client.sh } starten. Hier sollten Sie dem Admin-User ein Password spendieren.
\end{enumerate}
%
%
\subsection{Installation der Komponenten unter MS Windows}
%
%
\subsubsection{Die Installation der freien XML:DB eXist}
eXist ist eine Frei verf�gbare XML:BD, welche die entsprechenden Interfaces implementiert. F�r MyCoRe wurde ein auf diesen Schnittstellen basierender Persitence-Layer implementiert. So sollte die Nutzung von eXist direkt m�glich sein.
\begin{enumerate}
\item Download der aktuellen Version von eXist von \url{http://exist-db.org/}. Achtung, da dieses produkt noch stark im Wachsen ist, sollte hier die letzte Version geholt werden. Zum Test kam Version eXist-0.9.2.
\item entpacken Sie die Distribution in ein entsprechendes Verzeichnis.
\item Entfernen Sie zur Nutzung des stand-alone-Servers den Kommentar aus der Zeile {/tt uri=xmldb:exist://localhost:8081} im File {\it <installdir>\\eXist-0.9.2\\client.properties}.
\item Starten Sie den Server mit {\tt <installdir>\\eXist-0.9.2\\bin\\startup.bat }.
\item unter der URL \url{http://localhost:8080/exist/index.html} sollte jetzt die eXist-Homepage erscheinen. Dies nur als Test.
\item Wenn alles okay ist, starten Sie {\tt <installdir>\\eXist-0.9.2\\bin\\server.bat }.
\item Anschliessend k�nnen Sie auch den Client mit {\tt <installdir>\\eXist-0.9.2\\bin\\client.bat } starten. Hier sollten Sie dem Admin-User ein Password spendieren.
\end{enumerate}
%
%
