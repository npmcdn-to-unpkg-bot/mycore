%
%
\section{Hinweise zur Installation feier Datenbanken und XML:DB's}
%
%
\subsection{Installation der Komponenten unter Linux}
Die folgende Beschreibung erl�utert die Arbeit unter SuSE Linux, getestet wurde unter der Version 8.1. Sollten f�r RedHat Abweichungen auftreten, so werder diese gesondert vermerkt.
%
%
\subsubsection{Die Installation von MySQL}
\begin{enumerate}
\item Installieren Sie aus Ihrer Distribution die folgenden Pakete und danach ggf. noch vom Hersteller der Distribution per Netz angebotene Updates. Die angegebenen versionsnummern sind nur exemplarisch.
\begin{itemize}
\item {\bf mysql-3.23.52-83}
\item {\bf mysql-shared-3.23.52-83}
\item {\bf mysql-client-3.23.52-83}
\item {\bf mysql-devel-3.23.52-83}
\item {\bf mysql-bench-3.23.52-83}
\end{itemize}
\item Die Dokumentation steht nun unter {\it /usr/share/doc/package/mysql}.
\item Setzen Sie das Password f�r den User mysql und die Shell auf /bin/bash.
\item F�hre Sie das Kommando {\tt mysql\_install\_db } als mysql aus.
\item F�hre Sie das Kommando {\tt safe\_mysqld \& } als mysql aus.
\item Zum verifizieren, ob der Server l�uft nutzen Sie {\tt mysqladmin version} und  {\tt mysqladmin variables}.
\end{enumerate}
%
%
\subsection{Installation der Komponenten unter MS Windows}
%
%
