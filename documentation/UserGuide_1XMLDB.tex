%
%
\section{Hinweise zur Installation feier Datenbanken und XML:DB's}
%
%
\subsection{Installation der Komponenten unter Linux}
Die folgende Beschreibung erl�utert die Arbeit unter SuSE Linux, getestet wurde unter der Version 8.1. Sollten f�r RedHat Abweichungen auftreten, so werder diese gesondert vermerkt.
%
%
\subsubsection{Vorbereitung}
%
%
Als erster Schritt sind der Java-Compiler J2SE 1.3.1 oder h�her und ANT zu installieren. Beide Befinden sich unter SuSE in der Distribution.
\begin{itemize}
\item {\bf java2-jre-1.3.1-521} oder h�her
\item {\bf java2-1.3.1-521} oder h�her
\item {\bf jakarta-ant-1.5-76} oder h�her
\end{itemize}
Sinnvollerweise sollten Sie sich einen Benutzer (z. B. {\bf mcradmin} anlegen, unter welchem Sie dann Ihre Applikationen laufen lassen.
%
%
\subsubsection{Die Installation von MySQL}
\begin{enumerate}
\item Installieren Sie aus Ihrer Distribution die folgenden Pakete und danach ggf. noch vom Hersteller der Distribution per Netz angebotene Updates. Die angegebenen versionsnummern sind nur exemplarisch.
\begin{itemize}
\item {\bf mysql-3.23.52-83} oder h�her
\item {\bf mysql-shared-3.23.52-83} oder h�her
\item {\bf mysql-client-3.23.52-83} oder h�her
\item {\bf mysql-devel-3.23.52-83} oder h�her
\item {\bf mysql-bench-3.23.52-83} oder h�her
\end{itemize}
\item Die Dokumentation steht nun unter {\it /usr/share/doc/package/mysql}.
\item F�hre Sie das Kommando {\tt mysql\_install\_db } als mysql aus.
\item F�hre Sie das Kommando {\tt safe\_mysqld \& } als mysql aus.
\item Zum verifizieren, ob der Server l�uft nutzen Sie {\tt mysqladmin version} und  {\tt mysqladmin variables}.
\item Setzen Sie nun das {\bf root} Password mit dem Kommando {\tt /usr/bin/mysqladmin -u root  password PASSWORD} und {\tt mysqladmin -u root -h HOST password PASSWORD} . Ist das Password einmal gesetzt, m�ssen Sie zus�tzlich die Option -p verwenden.
\item die folgende Sequenz sorgt daf�r, dass der MyCoRe-User {\bf mcradmin} alle Rechte auf der Datenbank hat. Dabei werden bei der Ausf�hrung von Kommandos von {\bf localhost} aus keine Pssworte abgefragt. Von anderen Hosts aus muss 'ein\_password' eingegeben werden.
{\tt
\begin{verbatim}
mysql --user=root -pPASSWORD mysql
GRANT ALL PRIVILEGES ON *.* TO mcradmin@localhost WITH GRANT OPTION;
GRANT ALL PRIVILEGES ON *.* TO mcradmin@"%" IDENTIFIED BY 'ein\_passwort' WITH GRANT OPTION;
quit
\end{verbatim}
}
\item jetzt k�nnen Sie die Datenbasis f�r MyCoRe mit nachstehendem Kommando anlegen.
{\tt
\begin{verbatim}
mysqladmin -u mcradmin create mycore 
\end{verbatim}
}
\item Falls weitere Benutzer noch das Recht auf Selects von allen Hosts aus haben sollen, verwenden Sie die Kommanos
{\tt
\begin{verbatim}
mysql --user=mcradmin -pPASSWORD mycore
GRANT SELECT ON mycore.* TO mycorenutzer@'%';
quit
\end{verbatim}
}
\end{enumerate}
%
%
\subsection{Installation der Komponenten unter MS Windows}
%
%
