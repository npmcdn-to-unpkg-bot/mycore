\documentclass[a4paper,12pt]{book}
\usepackage{german}
\usepackage{fancyheadings}
\pagestyle{fancyplain}
\addtolength{\headwidth}{\marginparsep}
\addtolength{\headwidth}{\marginparwidth}
\renewcommand{\chaptermark}[1]%
      {\markboth{#1}{}}
\renewcommand{\sectionmark}[1]%
      {\markright{\thesection\ #1}}
\lhead[\fancyplain{}{\bfseries\thepage}]%
      {\fancyplain{}{\bfseries\rightmark}}
\rhead[\fancyplain{}{\bfseries\leftmark}]%
      {\fancyplain{}{\bfseries\thepage}}
\cfoot{}
\begin{document}
%-------   Vorspann
\title{MyCoRe User Guide}
\author{
    Frank L"utzenkirchen\\
    Jens Kupferschmidt\\
    Detlef Degenhardt\\
    Ulrike Kr"onert}
\maketitle
\setcounter{secnumdepth}{10}
\chapter*{Vorwort}
In diesem Dokument sind alle Arbeiten zum Start der Beispielanwendung und zur Gestaltung eigener Anwendungen beschrieben. Teilweise wird auch auf das MyCoRe Design Guide verwiesen.
\tableofcontents
\listoffigures
\listoftables
%-------    Hauptteil
\chapter{Voraussetzungen f"ur eine MyCoRe Anwendung}
Vor- und Nachteile der einzelnen Basissysteme [IBM / Linux / Sun] 
\section{Hinweise zur Bereitstellung der einzelnen Systeme (Quick Installation der wichtigen Komponenten und Test)}
\section{Hinweise zur Nutzung m"oglicher Webserver und Servlet-Maschinen und deren Konfiguration}
\chapter{Download und Installation des MyCoRe Kerns}
\section{Konfiguration zum "Ubersetzten des Kerns}
\section{Compile}
\chapter{Die MyCore Beispielanwendung}
\section{Grundlegender Aufbau und Ziel der Beispielanwendung}
\section{Vereinfachte Funktionsprinzipien der Anwendung}
\subsection{User- und Rechtesystem}
\subsection{Klassifikationen}
\subsection{Metadatenmodel}
\subsection{IFS und Content Store}
\subsection{Datenpr"asentation}
\subsection{Interaktive Arbeit mit den Daten}
\section{Download der Beispielanwendung}
\section{Konfiguration zur Arbeit mit den Beispieldaten}
Wenn man nur sections und subsections aneinanderf"ugt, wird offensichtlich kein Seitenumbruch gemacht??
\subsection{Grundlegende Konfigurationen (JDBC, CM, Logger, usw.)}
\subsection{Das User- und Rechtesystem  + Laden der Beispieldaten}
\subsection{Das Klassifikationsmodel + Laden der Beispieldaten}
\subsection{Das Metadatenmodel + Laden der Beispieldaten}
\subsection{Das IFS + Laden der Beispieldaten}
\section{Arbeiten mit der Kommandzeilen-Shell mycore.sh}
\subsection{"Ubersicht der Kommando und Beispiele}
\section{Arbeiten mit der Web-Anwendung}
\subsection{Konfiguration und Start der Webanwendung }
\subsubsection{Apache / Tomcat}
\subsubsection{Apache / Websphere}
\subsection{Das Layout-Setvlet und das Zusammenspiel der Servlets untereinander}
\subsection{Die Nutzung des Editor-Servlets}
\section{Zusammenarbeit mit anderen Installationen (Remote)}
\section{Vom Sample zum eigenen Dokumentserver}
Hier sollen Anpassungsschritte detailliert erkl"art werden.
\subsection{Anpassungen des Layout an eigene Bed�rfnisse}
\subsection{Weitere User und Gruppen / Nutzung zur selektierten Darstellung durch Stylesheets}
\subsection{Metadatenvererbung}
\subsection{Nutzung der OAI Schnittstelle }
\subsection{Arbeiten mit NBN}
\subsection{M"ogliche Workflow-Szenarien im Bibliotheksumfeld}
\section{Einbindung weiterer Content Stores (Helix \& Co.)}
\section{Hints \& Tips / Trobleshooting}
\chapter{Erstellen einer eigenen Anwendung auf Basis des MyCoRe-Kernes}
(am Beispiel des Papyrus-Projektes)
\section{Erforderliche Schritte zu einer Sammlungsanwendung mit eigenen Metadaten}
\begin{flushright}
{\em macht Jens}
\end{flushright}
\chapter{Weitere Anwendungen (Archivl"osung)}
\end{document}
