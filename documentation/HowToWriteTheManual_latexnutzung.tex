\chapter{Die Arbeit mit einem \LaTeX -- System}
\section{Die Installation}
\subsection{Die Installation unter SuSE Linux}
Die Linux-Distribution von SuSE bietet unter 8.x ein recht vollst�ndiges, umfangreiches und aktuelles \LaTeX -- System inklusive einer ganzen Reihe von zus�tzlichen Komponenten.
Folgende Pakete werden empfohlen zu installieren:\\
\begin{itemize}
\item {\bf te\_latex} - Alles zu und um \LaTeX
\item {\bf te\_pdf} - Eine TeX/\LaTeX - Version mit PDF-Ausgabeformat
\item {\bf tex4ht} - Ein TeX/\LaTeX Umwandler
\item {\bf ImageMagick} - Ein m�chtiges Bildverarbeitungs-Tool
\end{itemize}
\section{Verarbeitungskommandos}
\subsection{Unix-Kommandos}
Das Kommando {\tt latex {\it filename.tex}} erzeugt aus den \LaTeX - Quelle ein {\bf .dvi} - File, welches dann entsprechend weiterverarbeitet werden kann.\\[2ex]
Das Kommando {\tt dvips {\it filename.dvi}} generiert aus dem {\bf .dvi} - File ein Postscript-File mit der Endung {\bf .ps}.\\[2ex]
Um mit einem Rutsch direkt PDF-Dokumente zu erzeugen gen�gt das Kommando {\tt pdflatex {\it filename.tex}}.\\[2ex] 
Zus�tzlich haben wir ein Shell Script {\tt convert\_jpeg.sh } unter {\it \$MYCORE\_HOME/bin } gestellt, welches auf Basis von {\bf ImageMagick} von allen jpeg-Files Bounding Boxes zur Verarbeitung unter \LaTeX erzeugt.
