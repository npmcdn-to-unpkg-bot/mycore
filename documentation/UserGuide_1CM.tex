%
%
\section{Hinweise zur Installation des IBM Content Manager 8.2}
%
%
\subsection{Der IBM Content Manager unter AIX}
An dieser Stelle soll eine Kurzbeschreibung der Installation des IBM Content Managers 8.2 f"ur AIX von Holger K"onig, IBM Deutschland GmbH, wiedergegeben werden. \\[2ex]
\subsubsection{Vorbereitung}
\begin{enumerate}
\item Installieren Sie das AIX Betriebssystem mit dem Release 4.3.3 ML 10, 5.1 ML 01 oder 5.2.
\item Sorgen Sie daf"ur, dass 'Cultural Conversion' und 'Language' auf English US eingestellt ist.
\item Aktivieren Sie die Netzanbindung inklusive DNS.
\item F"ur die Betriebssystem-Releases 4.3.3 und 5.1 muss Java 1.3.1 entsprechend der Anleitung installiert werden. Erweitern Sie in {\it /etc/environment }  die {\it PATH} Variable um {\it /usr/java131/jre/bin} und {\it /usr/java131/bin}. Wenn Sie auch das Paket {\bf Java131.ext.java3d } mit installieren wollen, m�ssen Sie vorher die Pakete {\bf OpenGL.OpenGL\_X.adt} und {\bf OpenGL.OpenGL\_X.rte} installiert haben.
\item Installieren Sie den VAC Compiler Version 5.x oder 6.0 entsprechend der Anleitung und tragen Sie den Suchpfad unter {\it PATH} im File {\it /etc/environment } mit ein.
\item Aktivieren Sie das Lizenzsystem {\bf ifor} und tragen Sie sie Compilerlizenzen ein.
\end{enumerate}
\subsubsection{DB2}
\begin{enumerate}
\item Kopieren Sie das File {\it ese.sbcs.tar.Z} von der CD {\bf 'DB2 8.1 with FP1'} und entpacken Sie dieses.
\item {\tt ./db2setup}
\item W"ahlen Sie {\bf Install Products} $\rightarrow$ {\bf DB2 UDB Enterprise Server Edition}. Folgen Sie den Schritten:
\begin{itemize}
\item {\bf Netx}
\item {\bf Accept License}
\item Auswahl 'Custom' $\rightarrow$ {\bf Next}
\item Auswahl 'Install DB2 UDB Enterprise Server Edition on this computer' $\rightarrow$ {\bf Next}
\item Standartwerte lassn, 'Appliction Development Tools' zus"atzlich ausw"ahlen
\item Sprache 'Englisch' beibehalten $\rightarrow$ {\bf Next}
\item DAS User : Standartwert {\bf db2as} wenn m�glich beibehalten, Password setzen $\rightarrow$ {\bf Next}
\item Erzeugen der DB2 Instanz durch Auswahl 'Create a DB2 instance - 32 bit' $\rightarrow$ {\bf Next}
\item Auswahl 'Single-partition instance' $\rightarrow$ {\bf Next}
\item Eintrag des DB2 Instance owner : Standartwert ({\bf db2inst1}) m"oglichst lassen, Password setzten $\rightarrow$ {\bf Next} \footnote{Achten Sie darauf, keine exotischen Sonderzeichen zu nehmen, das macht im CM Probleme!}
\item Eintrag des DB2 Fenced users : Standartwert ({\bf db2fenc1}) m"oglichst lassen, Password setzten $\rightarrow$ {\bf Next} \footnote{Achten Sie darauf, keine exotischen Sonderzeichen zu nehmen, das macht im CM Probleme!}
\item Instance TCPIP : Auswahl 'Configure' $\rightarrow$ Service Name : db2c\_dv2inst1 $\rightarrow$ Port 50000 $\rightarrow$ {\bf Next}
\item Instance properties $\rightarrow$ Authentication Type : Server $\rightarrow$ beibehalten 'Autostart the instance at system startup' $\rightarrow$ {\bf Next}
\item Prepare the DB2 tools catalog  beibehalten 'Do not prepare the DB2 tools catalog on this computer' $\rightarrow$ {\bf Next}
\item Administrator contact : Standart beibehalten $\rightarrow$ {\bf Next} \footnote{Warnung ignorieren}
\item Contact : Standart beibehalten ({\bf db2inst1}) $\rightarrow$ {\bf Next}
\item Summary $\rightarrow$ {\bf Finish}
\item Warten (dauert etwas)
\item Setup complete $\rightarrow$ {\bf Finish}
\end{itemize}
\item Test der Installation:\\
\qquad\qquad{\tt su - db2inst1} \\
\qquad\qquad{\tt db2stop} \\
\qquad\qquad{\tt db2start} \footnote{Es sollte keine Nachricht bez"uglich der Lizenz erscheinen.}\\
\qquad\qquad{\tt db2level} \\
\end{enumerate}
\subsubsection{NSE}
%

%
%
\subsection{Der IBM Content Manager unter Solaris}

%
%
\subsection{Der IBM Content Manager unter Windows}

%
%

