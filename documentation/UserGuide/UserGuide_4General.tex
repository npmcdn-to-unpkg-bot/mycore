%
%
\section{Allgemeines}
%
%
Nachdem das Sample bei nun l�uft und erste Erfahrungen damit gesammelt wurden, soll nun auf dieser Grundlage aufbauend ein produktiver Dokumentserver aufgesetzt werden. Hier gilt es nun, das MyCoRe Modell in eine praxisorientierte Anwendung umzusetzen. Sicher werden bei jedem Anwender weitere zus�tzliche Anforderungen auftreten, welche innerhalb des Samples oder der nachfolgenden Ausf�hrungen keine Beachtung fanden. Die MyCoRe-Gruppe ist gern bereit, allgemeing�ltige Erg�nzungen in das Projekt mit aufzunehmen.\\[2ex]
F�r die Erstellung einer Dokument- und Multimedia-Server-Anwendung sind im groben die folgenden Schritte erforderlich. Modifizieren Sie das Sample schrittweise hin zu Ihrer eigenen Applikation. Pr�fen Sie in zwischenschritten immer, dass die gew�nschte Funktionalit�t noch erhalten geblieben ist.
\begin{itemize}
\item Kopieren des MyCoRe-Samples in einen gesonderten Verzeichniszweig (z. B. \mcrfile{/docserv}. Es ist von Vorteil diesen Pfad auch in einer environment variable abzulegen (z. B. \mcrcommand{export DOCSERV\_HOME=/docserv}).
\item Legen Sie f�r Ihre Anwendung die Namen f�r die Datenbank-Tabellen, XML- und IFS-Stores fest und �ndern Sie die Konfigurations-Dateien entsprechend. Erzeugen Sie nun die Data Stores und XML-Schema Files.
\item Modifizieren Sie die User-Daten entsprechend ihren Anforderungen und laden Sie das User-System.
\item Erstellen Sie sich alle ben�tigten Klassifikationen (z. B. eine aller Ihrer Einrichtungen) und laden Sie diese.
\item Laden Sie nun erste, per Hand oder Script erstellte Metadaten-Test-Files und laden Sie auch diese. Nun k�nnen Sie auf Commandline-Ebene schon mittels mycore.sh bzw. mycore.cmd Anfragen an das System stellen und erste Test durchf�hren.
\item Legen Sie die Stores f�r die Multimedia-Objekte fest, konfigureieren Sie diese und laden Sie einige Beispiele zum Test.
\item Legen Sie nun eine URL f�r ihren Server fest und setzen Sie einen Web-Server (WebSphere oder Apache/Tomcat) auf.
\item Installiern Sie Ihre Anwendung im Web-Server und modifizieren Sie schrittweise die Pr�sentation nach Ihren Bed�rfnissen.
\item Testen und integrieren Sie die in diesem Kapitel beschriebenen weiterff�hrenden Funktionalit�ten, welche nicht im MyCoRe-Sample enthalten sind.
\end{itemize}
Habe Sie all die Schritte bew�ltigt, sollte Ihnen nun ein ansprechender Dokument-Server zur Verf�gung stehen. Sollten Sie andere Applikationen, wie Sammlungen usw. aufbauen wollen, konsultieren Sie bitte auch das ProgrammerGuide, wo auf derartige MyCoRe-Erweiterungen n�her eingegangen wir.\\[2ex]

