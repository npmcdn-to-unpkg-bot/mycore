%
%
\subsection{Metadatenmodel}
%
%
Die zu speichenden Daten des Beispiels wie auch anderer vom Anwender entwickelter Applikationen teilen sich in unserem Modell in Metadaten und digitale Objkete. Hinsichtlich letzterer sei auf den Anschnitt {\bf 'IFS und Content Store'} verwiesen. Unter Metadaten verstehen wir in MyCoRe  alle beschreibenden Daten des Objektes, die extern hinzugef�gt, separat gespeichert und gesucht werden k�nnen. Dem gegen�ber stehen Daten welche die digitalen Objekte selbst mitbringen. In diesem Abschnitt werden nur erstere behandelt. \\[2ex]
Um die Metadaten besser auf unterschiedlichen Datenspiechern ablegen zu k�nnen, wurde eins System von XML-Strukturen entwickelt, welches es gestattet, neben den eigentlichen Daten wie Titel, Autor usw. auch Struktur- und Service-Informationen mit abzulegen. Die eigentlichen Nutzerdaten sind wiederum typisiert, was denen speicherunabh�ngige Aufzeichnung erheblich vereinfacht. Es steht dem Entwickler einer Anwendung jedoch frei, hier bei bedarf weitere hinzuzuf�gen. Im Folgenden soll nun der Aufbau der Metadaten-Objekte im Detail beschrieben werden. Zum verst�ndnis des Samples sei hier auch auf den vorigen Abschnitt verwiesen.\\[2ex]
Die Metadaten werden komplett in XML erfasst und verarbeitet. F�r die Grundstrukturen und Standard-Metadaten-Typen werden seitens MyCoRe bereits XMLSchema-Dateien mitgeliefert.
%
%
\subsubsection{Konkreter Aufbau eines Matadaten-Objektes}
\begin{center}
%\begin{boxedminipage}[t]{13cm}
%\bottomcaption{Allgemeiner Aufbau des Metadaten-Objektes}

\lstset{language=XML,fancyvrb=true,breaklines}
\begin{lstlisting}[frame=single]
<?xml version="1.0" cncoding="iso-8859" ?>
<mycoreobject
 xmlns:xsi="http://www.w3.org/2001/XMLSchema-instance"
 xsi:noNamespaceSchemaLocation="....xsd"
 xmlns:xlink="http://www.w3.org/1999/xlink"
 ID="..."
 label="..."
 >
 <structure>
  ...
 </structure>
 <metadata xml:lang="de">
  ...
 </metadata>
 <service>
  ...
 </service>
</mycoreobject>
\end{lstlisting}
%\end{boxedminipage}
\end{center}

F�r {\bf xsi:noNamespaceSchemaLocation} ist das entsprechende XMLSchema-File des Metadaten-Types anzugeben (z. B. \mcrfile{Document.xsd})\\[2ex]
Die {\bf ID} ist die eindeutige MCRObjectID.\\[2ex]
Der {\bf label} ist ein kurzer Text-String, der bei administrativen Arbeiten an der Datenbasis das Identifizieren einzelner Datens�tze erleichtern soll. Er kann maximal 256 Zeichen lang sein.\\[2ex]
Innerhalb der XML-Datenstruktur gibt es die Abschnitte {\bf structure}, {\bf metadata} und {\bf service} zur Trennung von Struktur-, Beschreibungs- und Wartungsdaten. Diese Tag-Namen sind reserviert und {\bf d�rfen NICHT anderweitig verwendet werden!}\\[2ex]
%
%
\subsubsection{Aufbau des XML-Knotens structure}
Im XML-Knoten {\bf structure} sind alle Informationen �ber die Beziehung des Metadaten-Objektes zu anderen Objekten abgelegt. Es werden derzeit die folgenden XML-Daten unter diesem Knoten abgelegt. Die Tag-Namen {\bf parents/parent}, {\bf children/child} und {\bf derobjects/derobject} sind reserviert und {\bf d�rfen NICHT anderweitig verwendet werden!} Alle Sub-Knoten haben einen Aufbau wie f�r MCRMetaLinkID beschieben.\\[2ex]
In {\bf parents} wird ein Link zu einem Eltern-Objekt gespeichert, sofern das referentierende Objekt Eltern hat. Ob dies der Fall ist, bestimmt die Anwendung. Das Tag dient der getsaltung von Vererbungsb�umen und kann durch den Anweder festgelegt werden. Siehe auch 'Programmers Guide', Abschnitt Vererbung. Die Werte f�r {\it xlink:title} und {\it xlink:label} werden beim Laden der Daten automatisch erg�nzt.\\[2ex]
Die Informationen �ber die {\bf children} hingegen werden durch das MyCoRe-System beim Laden der Daten {\bf automatisch} erzeugt und {\bf d�rfen nicht per Hand ge�ndert werden}, das sonst das Gesamtsystem nicht mehr konsistent ist. Werden die Metadaten eines Kindes oder eines Baumes von Kindern gel�scht, so wird in diesem Teil des XML-Knotens der Eintrag durch die Software entfernt.\\[2ex]
Das selbe gilt auch f�r den XML-Unterknoten {\bf derobjects}. In diesem Bereich werden alle Verweise auf die an das Metadaten-Objekt angehangenen digitalen Objekte gespeichert. Jeder Eintrag verweist mittels einer Referenz auf ein Datenobjekt vom Typ {\it mycorederivate}, wie es im nachfolgenden Abschnitt {\bf 'IFS und Content Store'} n�her erl�utert ist.
\begin{center}
%\begin{boxedminipage}[t]{13cm}
%\bottomcaption{Allgemeiner Aufbau des structure XML-Knotens}
\lstset{language=XML,fancyvrb=true,frame=btlr,breaklines,prebreak={\space\MyHookSign}}
\begin{lstlisting}{frame=single}
<structure>
 <parents class="MCRMetaLinkID">
  <parent xlink:type="locator" xlink:href="...mcr_id..." />
 </parents>
 <children class="MCRMetaLinkID">
  <child xlink:type="locator" xlink:href="...mcr_id..." xlink:label="..." xlink:title="..." />
  ...
 </children>
 <derobjects class="MCRMetaLinkID">
  <derobject xlink:type="locator" xlink:href="...mcr_id..." xlink:label="..." xlink:title="..." />
  ...
 </derobjects>
</structure>
\end{lstlisting}
%\end{boxedminipage}
\end{center}


