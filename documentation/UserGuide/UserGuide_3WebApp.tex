\section{Erzeugen und Konfigurieren der Web-Anwendung}
\subsection{Erzeugen der Web-Anwendung}
Durch Eingabe von
\begin{center} 
{\tt build.sh webapp } \qquad bzw. \qquad {\tt build.cmd webapp } 
\end{center} 
wird die MyCoRe Sample Web Application im Verzeichnis {\tt webapps}
erzeugt. Alternativ k�nnen Sie auch ein Web Application Archive (war)
erzeugen, indem Sie
\begin{center}
{\tt build.sh war } \qquad bzw. \qquad {\tt build.cmd war }
\end{center}
aufrufen. 

Das MyCoRe Build-Script kopiert beim Erzeugen der Web Applikation
auch alle externen, erforderlichen jar-Dateien Ihrer verwendeten 
Datenbank-Systeme (IBM Content Manager / DB2, MySQL, eXist) in das
Verzeichnis {\tt WEB-INF/lib}, entsprechend den Vorgaben Ihrer
Konfiguration in {\tt build.properties}. Beachten Sie dazu bitte die 
Hinweise in der Ausgabe beim Erzeugen der Web Application.

\subsection{Konfiguration des Web Application Server}

\subsubsection{ Tomcat}
Die grundlegende Installation von Tomcat wurde bereits beschrieben. Nun soll auf dieser Basis das die WEB-Anwendung des MyCoRe-Samples installiert werden. Dabei ist an dieser Stelle nur ein einfaches Szenario auf der Basis der Tomcat-Grundinstallation beschrieben. F�r die Konfiguration komplexerer Modelle, z. B. mehrere Applikationen nebeneinander, gibt es weiter hinten in diesem Dokument eine ausf�hrliche Anleitung.\\[2ex]
Folgende Schritte sind auszuf�hren:
\begin{enumerate}
\item \mcrcommand{su -}
\item \mcrcommand{cd \$CATALINA\_HOME/webapps}
\item \mcrcommand{cp \$MYCORE\_SAMPLE\_HOME/mycoresample.war .}
\item \mcrcommand{rctomcat restart}
\item \mcrcommand{rm mycoresample/WEB-INF/lib/xerces*}
\item \mcrcommand{rm mycoresample/WEB-INF/lib/xalan*}
\item \mcrcommand{rctomcat restart}
\end{enumerate}
Nun sollten Sie auf die Beispielanwendung mit der URL \url{http://localhost:8080/mycoresample} zugreifen k�nnen. Testen Sie nun die Anwendung!

\subsubsection{ Websphere}
Im Kapitel 1 wurde ja bereits beschrieben, wie die Anwendung IBM WebSphere zu installieren ist. Diese soll als Servlet-Engine zur Anwendung kommen, wenn der IBM Content Manager 8 als Persitence-Layer verwendet wird. Die Konfiguration von WebSphere erfolgt via Web-Anwendung. Starten Sie dazu den Adminserver mittels
\begin{center}
\mcrcommand{/usr/WebSphere/AppServer/bin/startServer.sh server1}
\end{center}
�ffnen Sie nun eine Web-Browser mit der URL \url{http://<hostname>:9090/admin} und melden Sie sich an.\\[2ex]
Nun sind folgende Schritte durchzuf�hren:
\begin{enumerate}
\item (linke Seite) {\bf Server} $\rightarrow$ {\bf Application Server}
\item (rechte Seite) {\bf NEW}
\item (rechte Seite) Server Name $\rightarrow$ {\bf mycoresample} $\rightarrow$ {\bf NEXT}
\item (rechte Seite) {\bf FINISH}
\item (linke Seite) {\bf Applications} $\rightarrow$ {\bf Install New Applications}
\item (rechte Seite) {\bf Server Path} $\rightarrow$ Pfad zum File \mcrfile{mycoresample.war} eintagen $\rightarrow$ {\bf /mycoresample} im Feld Context Root eintragen $\rightarrow$ {\bf NEXT}
\item (rechte Seite) Preparing for application installation $\rightarrow$ {\bf NEXT}
\item (rechte Seite) Step 1 $\rightarrow$ {\bf NEXT}
\item (rechte Seite) Step 2 $\rightarrow$ {\bf NEXT}
\item (rechte Seite) Step 3 $\rightarrow$ Ausw�hlen {\bf mycoresample} in der Checkbox $\rightarrow$ dann ausw�hlen der Zeile mit {\bf server=mycoresample} $\rightarrow$ {\bf APPLY}
\item (rechte Seite) Step 3 $\rightarrow$ {\bf NEXT}
\item (rechte Seite) Step 4 $\rightarrow$ {\bf FINISH}
\item (linke Seite) {\bf Server} $\rightarrow$ {\bf Application Server}
\item (rechte Seite) {\bf mycoresample} $\rightarrow$ {\bf Process Definition} $\rightarrow$ {\bf Process Execution} 
\item (rechte Seite) User auf {\bf mcradmin} setzen
\item (rechte Seite) Group auf {\bf mcr} setzen
\item (rechte Seite) {\bf APPLY} $\rightarrow$ {\bf OK}
\item (rechte Seite) {\bf Java Virtual Machine} $\rightarrow$ {\bf Classpath}
\item (rechte Seite) einf�gen der Pfade \\ /home/db2inst1/sqllib/java12/db2java.zip: \\ /usr/lpp/cmb/lib/cmbsdk81.jar: \\ /usr/lpp/cmb/cmgmt
\item (rechte Seite) {\bf APPLY} $\rightarrow$ {\bf OK}
\item (rechte Seite) oben auf den Text {\bf save} klicken
\item (rechte Seite) {\bf SAVE}
\item (linke Seite) {\bf Environment} $\rightarrow$ {\bf Update Web Server Plugin} $\rightarrow$ {\bf OK}
\item {\bf Logout}
\end{enumerate}
Nun muss der Application Server gestartet werden:
\begin{center}
\mcrcommand{/usr/WebSphere/AppServer/bin/startServer.sh mycoresample}
\end{center}
Da die Anwendung als {\bf mcradmin} ausgef�hrt wird, kommt es zu einem Schreibfehler in den Log-Files. Hier ist nun folgendes zu tun:
\begin{enumerate}
\item \mcrcommand{chown -R mcradmin:mcr /usr/WebSphere/AppServer/logs/mycoresample}
\item \mcrcommand{chmod 666 /usr/WebSphere/AppServer/logs/activity.log}
\item \mcrcommand{cd /usr/WebSphere/AppServer/temp/<hostname>}
\item \mcrcommand{mkdir mycoresample}
\item \mcrcommand{chown -R mcradmin:mcr mycoresample}
\end{enumerate}
Danach ist der Server nocheinmal zu stoppen und neu zu starten:
\begin{center}
\mcrcommand{/usr/WebSphere/AppServer/bin/stopServer.sh mycoresample}
\mcrcommand{/usr/WebSphere/AppServer/bin/startServer.sh mycoresample}
\end{center}
Jetzt sollten Sie auf das MyCoRe-Sample unter der URL \url{http://<hostname>/mycoresample} zugreifen k�nnen.

\subsection{Die Nutzung des Editor-Servlets}

