\section{Erzeugen und Konfigurieren der Web-Anwendung}
\subsection{Erzeugen der Web-Anwendung}
Durch Eingabe von
\begin{center} 
{\tt build.sh webapp } \qquad bzw. \qquad {\tt build.cmd webapp } 
\end{center} 
wird die MyCoRe Sample Web Application im Verzeichnis {\tt webapps}
erzeugt. Alternativ k�nnen Sie auch ein Web Application Archive (war)
erzeugen, indem Sie
\begin{center}
{\tt build.sh war } \qquad bzw. \qquad {\tt build.cmd war }
\end{center}
aufrufen. 

Das MyCoRe Build-Script kopiert beim Erzeugen der Web Applikation
auch alle externen, erforderlichen jar-Dateien Ihrer verwendeten 
Datenbank-Systeme (IBM Content Manager / DB2, MySQL, eXist) in das
Verzeichnis {\tt WEB-INF/lib}, entsprechend den Vorgaben Ihrer
Konfiguration in {\tt build.properties}. Beachten Sie dazu bitte die 
Hinweise in der Ausgabe beim Erzeugen der Web Application.

\subsection{Konfiguration des Web Application Server}
\subsubsection{ Tomcat}
\subsubsection{ Websphere}
\subsection{Die Nutzung des Editor-Servlets}

