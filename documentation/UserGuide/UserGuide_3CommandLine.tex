\section{Arbeiten mit dem MyCoRe Command Line Interface}
\subsection{Erzeugen der Skripte mycore.sh / mycore.cmd}
Neben dem MyCoRe-Web-Interface kann f�r administrative 
Zwecke das MyCoRe Command Line Interface (CLI) genutzt werden.
Zum Aufruf des CLI m�ssen Sie zun�chst die erforderlichen Shell-Skripte erzeugen:
\begin{center} 
{\tt build.sh script } \qquad bzw. \qquad {\tt build.cmd script } 
\end{center} 
Dieser Aufruf generiert die Shell-Skripte {\tt bin/mycore.sh} (Unix) bzw. 
{\tt bin/mycore.cmd} (Windows).

\subsection{Aufruf des CommandLineInterface}
Starten Sie das MyCoRe Command Line Interface durch Aufruf von
{\tt bin/mycore.sh} (Unix) bzw. {\tt bin/mycore.cmd} (Windows).
Sie erhalten eine �bersicht �ber die verf�gbaren Befehle durch
Eingabe von
\begin{center}
{\tt help }
\end{center}
Sie verlassen das CommandLineInterface durch Eingabe von
\begin{center}
{\tt quit } \qquad oder \qquad {\tt exit } 
\end{center}

\subsection{Tests auf der Basis des CommandLineInterface}
Nachdem Sie nun die Testdaten geladen haben, besteht schon einmal die M�glichkeit, zu testen, ob die geladenen Daten sich anfassen lassen. Hierzu k�nnen sie das ConnamdLineInterface benutzen. \\[2ex]
Starten Sie {\tt bin/mycore.sh} (Unix) bzw.  {\tt bin/mycore.cmd} (Windows) und versuchen Sie zum Beispiel folgende Kommandos:
\begin{verbatim}
login gandalf
alleswirdgut
list all users
list all groups
list all privileges

query local class /mycoreclass[@ID="MyCoReDemoDC_class_0002"]

query local document /mycoreobject[@ID="MyCoReDemoDC_document_0001"]

save derivate MyCoReDemoDC_derivate_0002 to derivate_2
\end{verbatim}
