\subsection{Tomcat und Apache 2}
Ziel dieses Szenarios ist, dass unter mcradmin (oder einem anderen User) mehrere Applikationen mit verschiedenen URL Aliasen parallel laufen und sich dabei nicht ins Gehege kommen. Konkret bedeuted dies, dass verschiedene Tomcat Instanzen, die �ber einen Connector in den Apache Server integriert sind, parallel nebeneinander laufen. Zudem k�nnen die einzelenen Applikationen �ber eigene URLs angesprochen werden was mit Aapche �ber sogenannte virtual hosts realisiert wird.
Das folgende Beispiel ist etwas SuSE spezifisch sollte sich jedoch ohne weiteres auch auf eine andere Distribution �bertragen lassen . Eine sehr gute Anleitung findet sich jedoch auch in der Dokumentation von Tomcat.

\subsubsection{Funktionsweise}


\subsubsection{Installation}
Als Ausgangsbedingung wird die in Kapitel \ref{sec:installationlinux} beschriebene Installation zugrundegelegt.
In allen g�ngigen Distributionen ist auch der Tomcat Server enthalten. Unter SuSE empfehle ich das von mir neu erstellte Tomcat Paket zu installieren. Es sollte auch mit niedrigeren Version funkltionieren, allerding sind diese zum Teil noch etwas buggy. Ansonsten kann man auch die Binaries von der Tomcat Homepage herunterladen.
\begin{itemize}
\item {\bf jakarta-tomcat-4.1.27-37} oder h�her
\item {\bf apache2-jakarta-tomcat-connectors-4.1.27-37} oder h�her
\end{itemize}

\subsection{Einrichten der Tomcat Instanzen}
Zum Betreiben mehrer Tomcat Instanzen werden folgende zwei Variablen ben�tigt \$CATALINA\_BASE und \$CATALINA\_HOME.

\begin{quotation}
\$CATALINA\_HOME: This environment variable indicates the location of the root Tomcat install directory. This is where Tomcat can find the main scripts, system jar files, etc. As outlined in this document, the value should be \mcrpath{/opt/jakarta/tomcat}.
\$CATALINA\_BASE: When running multiple instances of Tomcat this variable indicates where the root location of one of the instances. This is so Tomcat can find instance configuration information, temporary file locations, etc. When you have a single instance of Tomcat it is typically the same as \$CATALINA\_HOME.
\end{quotation}

F�r eine Mehr-Instanzen Umgebung legen sie bitte folgendes Verzeichnis an:
F�hren sie folgende Befehle aus:
\mcrcommand{cd /dlwww}
\mcrcommand{mkdir tomcat-instances}
\mcrcommand{cd tomcat-instances}

Die Beschreibung ist f�r mehere Tomcat Instanzen geeignet, jedoch wird im Folgenden nur eine Instanz beschrieben. F�r weitere Instanten ist analog  dazu vorzugehen.
Als n�chstes wird f�r jede neue Tomcat Instanz ein Verzeichnis mit folgender Unterverzeichnisstruktur angelegt.

\mcrcommand{mkdir mycoresample}
\mcrcommand{cd mycoresample}

\mcrcommand{cp -r \$CATALINA\_HOME/conf .}
\mcrcommand{mkdir bin logs temp webapps work}

Kopiere die Datei \mcrfile{mycoresample.war} in das \mcrpath{\$CATALINA\_BASE/webapps} Verzeichnis.

Erstelle das mycoresample.sh Skript zum Starten und Stoppen in \mcrpath{\$CATALINA\_BASE/bin}.

\begin{verbatim}
#!/bin/sh
# set the environment

CATALINA\_BASE=/dlwww/tomcat-instances/mycoresample
CATALINA\_HOME=/opt/jakarta/tomcat
JAVA_HOME=/usr/lib/java
CATALINA\_OPTS="$CATALINA\_OPTS -server -Xms64m -Xmx256m -Xincgc"

#f�r IBM Java
#CATALINA\_OPTS="$CATALINA\_OPTS -Xms64m -Xmx256m -Xgcpolicy:optavgpause"


export CATALINA\_BASE CATALINE_HOME JAVA_HOME CATALINA\_OPTS

exec "$CATALINA\_HOME"/bin/catalina.sh "$@"
\end{verbatim}

Nun legen sie am besten die Datei \mcrfile{server.xml} neu im \mcrpath{\$CATALINA\_BASE/conf} mit folgendem Inhalt an.
�ndern sie den Hostnamen entsprechend ihrer Umgebung. Es ist der selbe Hostname wie wir ihn im folgenden Abschnitt unter VirtualHost einrichten. Wenn sie den Port �ndern wollen vergessen sie nicht auch Anpassungen in den Dateien \mcrfile{jk2.properties} und \mcrfile{workers2.properties} vorzunehmen.

\begin{verbatim}
<!-- Mycore Configuration File -->

<Server port="8006" shutdown="SHUTDOWN" debug="0">
  <!-- Define an Apache-Connector Service -->

  <Service name="Tomcat-Apache">

   <!-- Define a Coyote/JK2 AJP 1.3 Connector on port 8019 -->
    <Connector className="org.apache.coyote.tomcat4.CoyoteConnector"
               port="8019" minProcessors="5" maxProcessors="75"
               enableLookups="false"
               acceptCount="10" debug="0" connectionTimeout="0"
               useURIValidationHack="false"
               protocolHandlerClassName="org.apache.jk.server.JkCoyoteHandler"/>

      <Engine name="Apache" defaultHost="your_host_name" debug="0">

         <Logger className="org.apache.catalina.logger.FileLogger"
              prefix="apache_log." suffix=".txt"
              timestamp="true"/>

<!-- Access log processes all requests for this virtual host. -->
      <Valve className="org.apache.catalina.valves.AccessLogValve"
       directory="logs"  prefix="your_host_name" suffix=".txt"
       pattern="common" resolveHosts="false"/>

        <Host name="your_host_name" debug="0"
         appBase="/dlwww/tomcat-instances/mycoresample/webapps"
         unpackWARs="true" autoDeploy="true">

			<Alias>mycoresample</Alias>

			<Context path="" docBase="" debug="0"/>
            <Valve className="org.apache.catalina.valves.AccessLogValve"
             directory="logs"  prefix="home_access_log." suffix=".txt"
            pattern="common" resolveHosts="false"/>

		</Host>

   </Engine>

  </Service>

</Server>

\end{verbatim}

Um den Connector einzurichten sind folgende Dateien zu editieren \mcrfile{jk2.properties} und \mcrfile{workers2.properties}

Folgende zwei Zeilen in \mcrfile{\$CATALINA\_HOME/conf/jk2.properties} einf�gen.

\begin{verbatim}
handler.list=channelSocket,request
channelSocket.port=8019
\end{verbatim}

Am besten die Datei \mcrfile{\$CATALINA\_HOME/conf/workers2.properties} neu anlegen.

\begin{verbatim}
[logger]
level=DEBUG

# Alternate file logger
[logger.file:0]
level=DEBUG
file=/var/log/jk2.log

[config:]
file=/opt/jakarta/tomcat/conf/workers2.properties
debug=0
debugEnv=0

[shm:]
info=Scoreboard. Required for reconfiguration and status with multiprocess servers
file=/var/log/jk2.shm
size=1000000
debug=0
disabled=0

[workerEnv:]
info=Global server options
timing=1
debug=0

[lb:lb]
info=Default load balancer.
debug=0

[channel.socket:localhost:8019]
info=Ajp13 forwarding over socket
debug=0
group=lb
tomcatId=mycoresample-tc
lb_factor=1

[status:]
info=Status worker, displays runtime informations

[uri:/jkstatus/*]
info=Display status information and checks the config file for changes.
group=status:

\end{verbatim}

Der Apache Server wird �ber die \mcrfile{httpd.conf} folgenderma�en eingerichtet.

\begin{verbatim}


LoadModule jk2_module       /usr/lib/apache2/mod_jk2.so

<IfModule mod_jk2.c>
  JkSet config:file /opt/jakarta/tomcat/conf/workers2.properties
</IfModule>

<VirtualHost *>
     ServerAdmin admin@your_host_name
     DocumentRoot /dlwww/tomcat-instances/mycoresample/webapps
     ServerName your_host_name

Alias /mycoresample "/dlwww/tomcat-instances/mycoresample/webapps/mycoresample"
<Directory "/dlwww/tomcat-instances/mycoresample/webapps/mycoresample" >
  Options Indexes FollowSymLinks
  DirectoryIndex
  AddHandler jakarta-servlet2 .jsp
</Directory>

<Location "/mycoresample/WEB-INF" >
  AllowOverride None
  Deny from all
</Location>

<Location "/mycoresample/META-INF" >
  AllowOverride None
  Deny from all
</Location>

<Location "/mycoresample*.xml" >
  SetHandler jakarta-servlet2
  JkUriSet group lb
  JkUriSet servlet MCRStaticXMLFileServlet
  JkUriSet host mycoresample.feld45.lan
  JkUriSet context /mycoresample
</Location>

<Location "/mycoresample/servlets/MCRSearchMaskServlet" >
  SetHandler jakarta-servlet2
  JkUriSet group lb
  JkUriSet servlet MCRSearchMaskServlet
  JkUriSet host mycoresample.feld45.lan
  JkUriSet context /mycoresample
</Location>

<Location "/mycoresample/servlets/MCRQueryServlet" >
  SetHandler jakarta-servlet2
  JkUriSet group lb
  JkUriSet servlet MCRQueryServlet
  JkUriSet host mycoresample.feld45.lan
  JkUriSet context /mycoresample
</Location>

<Location "/mycoresample/servlets/MCRNBNResolver" >
  SetHandler jakarta-servlet2
  JkUriSet group lb
  JkUriSet servlet MCRNBNResolver
  JkUriSet host mycoresample.feld45.lan
  JkUriSet context /mycoresample
</Location>

<Location "/mycoresample/servlets/MCRNBNResolver/*" >
  SetHandler jakarta-servlet2
  JkUriSet group lb
  JkUriSet servlet MCRNBNResolver
  JkUriSet host mycoresample.feld45.lan
  JkUriSet context /mycoresample
</Location>

<Location "/mycoresample/servlets/MCRFileNodeServlet/*" >
  SetHandler jakarta-servlet2
  JkUriSet group lb
  JkUriSet servlet MCRFileNodeServlet
  JkUriSet host mycoresample.feld45.lan
  JkUriSet context /mycoresample
</Location>

<Location "/mycoresample/servlets/MCROAIDataProvider" >
  SetHandler jakarta-servlet2
  JkUriSet group lb
  JkUriSet servlet MCROAIDataProvider
  JkUriSet host mycoresample.feld45.lan
  JkUriSet context /mycoresample
</Location>

<Location "/mycoresample/servlets/MCRLoginServlet" >
  SetHandler jakarta-servlet2
  JkUriSet group lb
  JkUriSet servlet MCRLoginServlet
  JkUriSet host mycoresample.feld45.lan
  JkUriSet context /mycoresample
</Location>

</VirtualHost>


\end{verbatim}
Das sollte es auch schon gewesen sein. Jetzt nur noch starten mit \mcrcommand{\$CATALINA\_BASE/bin/mycoresample.sh} und \mcrcommand{rcapache2 restart}. Ein Blick auf \url{http://your_host_name/mycoresample} sollte jetzt die Startseite liefern. Zudem lohnt sich oft auch \url{http://your_host_name/jkstatus} aufzurufen um zu den Status des JK Connectors zu �berpr�fen.
Zum erstellen des Mappings habe ich das Java Progr�mmchen WebXml2Jk verwendet, das in den Tomcat-Connector Sources enthalten ist.\\[2ex]

