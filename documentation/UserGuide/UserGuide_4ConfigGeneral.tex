%
%
\subsection{Grundlegendes}
%
%
Um einen Dokument-Server oder andere MyCore-Anwendungen problemlos als
eigenst�ndige Anwendung zu betreiben, ist es notwendig hierf�r auch
separate Speicherbereiche f�r die Daten und Objekte bereitzustellen.
Hierzu muss die Konfiguration an verschiedenen Stellen ge�ndert werden.
Oft reicht schon die �nderung der Tabellennamen aus. F�r diese Beispiel
wird der Prefix {\bf DOC} verwendet.\\[ex2]
Im nachfolgenden Abschnitt wird davon ausgegangen, dass Sie den
MyCoRe-Sample Baum in ein gesondertes Anwendungsverzeichnis kopiert haben,
z. B. \mcrfile{~/docserv} welches durch die Environment Variable
z. B. \mcrcommand{export DOCSERV\_HOME=/docserv} beschrieben wird.\\[ex2]
An dieser Stelle sollen Sie auch nocheinmal darauf hingewiesen werden,
dass es auch unver�nderliche Einstellungen gibt, welche Sie beim 
compilieren des MyCoRe-Kernes einmal festgelegt haben. Die betrifft vor
allem die benutzte SQL-Datenbank (DB2, MySQL, ...) und den verwendeten 
Metadaten-Store (CM8, XML:DB,...). Diesen benutzen alle Ihre Applikationen
gemeinsam!\\[ex2]
Bearbeiten Sie folgende Konfigurationsstellen:

\begin{itemize}
\item {\bf Klassifikationen} - {\tt mycore.properties.classification}
\begin{verbatim}
# Configuration for SQL store
MCR.classifications_store_sql_table_class=DOCCLASS
MCR.classifications_store_sql_table_classlabel=DOCCLASSLABEL
MCR.classifications_store_sql_table_categ=DOCCATEG
MCR.classifications_store_sql_table_categlabel=DOCCATEGLABEL

# for classifications
MCR.linktable_store_sql_table_class=DOCLINKCLASS

# for links
MCR.linktable_store_sql_table_href=DOCLINKHREF
\end{verbatim}

\item {\bf Internal Filesystem} - {\tt mycore.properties.ifs}
\begin{verbatim}
# File metadata persistence implementation and table name to store nodes
MCR.IFS.FileMetadataStore.SQL.TableName=DOCFSNODES
\end{verbatim}

\item {\bf Usersystem} - {\tt mycore.properties.users}
\begin{verbatim}
# SQL tables for the user management
MCR.users_store_sql_table_users=DOCUSERS
MCR.users_store_sql_table_groups=DOCGROUPS
MCR.users_store_sql_table_group_members=DOCGROUPMEMBERS
MCR.users_store_sql_table_group_admins=DOCGROUPADMINS
MCR.users_store_sql_table_privileges=DOCPRIVS
MCR.users_store_sql_table_privs_lookup=DOCPRIVSLOOKUP
\end{verbatim}

\item {\bf Metdaten Store mit CM8} - {\tt mycore.properties.private}
Hier sind vor allem die richtigen Parameter f�r die Content Manager TextSearch
Anbindung einzutragen.

\begin{verbatim}
# Special values for the text search engine
MCR.persistence_cm8_textsearch_indexdir=/home/db2inst1/sqllib/db2ext/docindexes
MCR.persistence_cm8_textsearch_workingdir=/home/db2inst1/sqllib/db2ext/docindexes
\end{verbatim}

\item {\bf Metdaten Store mit CM8} - {\tt mycore.properties.cm8}
Weiterhin m�ssen noch die ItemType-Namen und die Prefixes f�r die
neue Anwendung festgelegt werden.
\begin{verbatim}
# Special values for the objects for CM8
MCR.persistence_cm8_document=DOC_Demo_Doc
MCR.persistence_cm8_document_prefix=ia
MCR.persistence_cm8_legalentity=DOC_Demo_Legal
MCR.persistence_cm8_legalentity_prefix=ib
MCR.persistence_cm8_derivate=DOC_Demo_Der
MCR.persistence_cm8_derivate_prefix=ic
\end{verbatim}

\item {\bf Metdaten Store mit eXist} - {\tt mycore.properties.private}
Nach dem Eintrag dieser Kondigurationszeile ist die eXist-Collection {\bf doc}
anzulegen. Unter dieser m�ssen dann die weiteren Collections analog zum
MyCoRe-Sample angelegt werden.

\begin{verbatim}
MCR.persistence_xmldb_database_url=xmldb:exist://localhost:8081/db/doc
\end{verbatim}

\end{itemize}

