%
%
\subsection{Aubau eines verteilten Dokument-Servers}
%
%
MyCoRe bietet die M�glichkeit, mehrere Server-Instanzen mit gleichen Datenmodellen zu einem gemeinsamen virtuellen Server zusammenzuschalten. Dies ist relativ leicht zu konfigurieren. Wichtig ist dabei jedoch, dass die Datenmodelle in ihren Strukturen und Tag-Bezeichnern identisch sind, da sonst die Suche nach Dokumenten oder Objekten erfolglos verl�uft.\\[2ex]
Die Anzahl der zusammengeschalteten Server ist theoretisch nicht begrenzt, in der Praxis sind mehr als 5-10 bestimmt wenig sinnvoll. Das System ist so gestaltet, dass beim Ausfall einzelner Instanzen die Arbeit mit den anderen nicht beeintr�chtigt wird. Es fehlen lediglich die Ergebnisse des ausgefallenen Servers. Bevor Sie mit der Konfiguration beginnen, sollten Sie den Abschnitt 'Die Zusammenarbeit mit anderen MyCoRe-Sample-Installationen' lesen.\\[2ex]
Um weitere Systeme zu erg�nzen oder deren Eintr�ge zu �nderer, gehen Sie wie folgend vor:\\[2ex]

Im Konfigurations-File \mcrfile{mycore.properties.private} unter \mcrfile{\$DOCSERV\_HOME/config} ist die Zeile mit {\bf MCR.remoteaccess\_hostaliases} anzupassen und im File \mcrfile{mycore.properties.remote} ist der Zugriff auf die neue Instanz zu definieren.
\begin{verbatim}
# List of the remote server aliases
MCR.remoteaccess_hostaliases=remote,NewRemoteSystem
\end{verbatim}

\begin{verbatim}
# Configuration for the new host with remote access
MCR.remoteaccess_NewRemoteSystem_query_class=org.mycore.backend.remote.MCRServletCommunication
MCR.remoteaccess_NewRemoteSystem_host=the.new.server.name
MCR.remoteaccess_NewRemoteSystem_protocol=http
MCR.remoteaccess_NewRemoteSystem_port=80
MCR.remoteaccess_NewRemoteSystem_query_servlet=/mycoresample/servlets/MCRQueryServlet
MCR.remoteaccess_NewRemoteSystem_ifs_servlet=/mycoresample/servlets/MCRFileNodeServlet
\end{verbatim}

Erg�nzen Sie nun noch die Suchmasken unter \mcrfile{\$DOCSERV\_HOME/config} um die weiteren Search-Instanzen und Testen Sie Ihre Konfiguration. Achtung, wenn Sie diesen Dienst anderen anbieten, muss Ihr Web-Servere aktiv sein!

