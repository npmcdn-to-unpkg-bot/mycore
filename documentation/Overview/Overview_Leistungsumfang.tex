MyCoRe ist ein Kernsystem, das alle Grundfunktionen einer digitalen Bibliothek abdeckt.
Es kann als Basis f"ur eigene Implementierungen verwendet werden, so dass
eigene spezialisierte L"osungen insbesondere hinsichtlich der Datenmodellierung
relativ schnell und einfach realisiert werden k"onnen.

Nachfolgend werden die Kernkomponenten mit einer Kurzbeschreibung der jeweiligen
Funktionalit"aten aufgef"uhrt.

\section{Dokumenten- und Personenmetadaten}
Das Metadatenmodell gew"ahrleistet die Mehrsprachigkeit aller relevanten Informationen.
Eine Reihe von Grundtypen f"ur die Metadatenmodellierung sind bereits im Kernsystem
implementiert.
Zu den angebotenen Grundtypen geh"oren.....

\section{Hierarchisches Klassifikationssystem}
\section{Internes Dateisystem}

\section{Verteilte Suche und Schnittstellen}
\subsection{Verteilte Suche}
\subsection{Schnittstellen zu OAI, Z39.50, Webservices}

\section{Benutzer- und Zugriffsrechteverwaltung}
\subsection{Benutzerverwaltung}
\label{sec:LeistungsumfangUsermanagement} 
Im Subsystem Benutzermanagement wird die Verwaltung derjenigen Personen geregelt,
die mit dem System umgehen (zum Beispiel als Autoren Dokumente einstellen). 
Dazu bietet die Benutzerverwaltung insbesondere die M"oglichkeit, dass sich Benutzer 
und Benutzerinnen am System authentifizieren k"onnen.

Die Gesch"aftsprozesse einer Benutzerverwaltung wie zum Beispiel das Anlegen neuer 
Benutzer, das Setzen von Passw"ortern, das Deaktivieren von Benutzern usw. erfordern 
unterschiedliche Privilegien. 
Das Benutzerverwaltungs-Komponente des MyCoRe-Projekts erm"oglicht eine detaillierte 
Vergabe von Privilegien auf der Basis von Benutzergruppen. 
Zur Entlastung der Betreiber der digitalen Bibliothek ist es insbesondere m"oglich, 
dediziert administrative Privilegien an Kunden zu delegieren. 
Dies wird durch das Konzept von Gruppenadministratoren erm"oglicht, die besondere
Rechte f"ur die Verwaltung der Gruppenmitglieder besitzen.
Durch ein System von Regeln wird dabei gew"ahrleistet, dass sich ein Benutzer oder 
eine Benutzerin nicht selbst h"ohere Privilegien zuweisen kann, als vom jeweils 
"ubergeordneten Administrator festgelegt wurde.

\subsection{Verwaltung von Zugriffsrechten}
Dieses Subsystem ist derzeit noch nicht implementiert.

\subsection{Digital Rights Management}
Dieses Subsystem ist derzeit noch nicht implementiert.

\section{Workflow- und Workbasketfunktionen}
Dieses Subsystem ist derzeit noch nicht implementiert.

\section{Benutzer- und Autoreninterface}



 