An deutschen Universit"aten werden zunehmend multimediale Daten sowohl f"ur
die Forschung als auch zur Unterst"utzung der Lehre in elektronischer
Form erstellt.
Beispiele hierf"ur sind Digitalisate von historischen Unikaten wie
altert"umlichen Handschriften, Partituren, Papyri oder M"unzen
aber auch Lehrvideos, Animationen und Simulationen.
Diese Best"ande m"ussen in einer geeigneten Umgebung systematisch
erschlossen und auch langfristig zug"anglich gemacht werden, eine
Aufgabe, die von einer digitalen Bibliothek geleistet werden kann.
Dabei werden die multimedialen Daten mit Metadaten versehen, um ein 
sp"ateres Auffinden oder Zusammenstellen bestimmer Dokumente zu erleichtern.

In diesem Kontext entstand auch Ende 1997 an der Universit"at Essen
das Projekt MILESS \footnote{http://miless.uni-essen.de} (Multimedialer
Lehr- und Lernserver Essen).
Dieses Open Source Projekt wurde zu einer solchen Reife gef"uhrt, dass
andere Universit"aten, die vor "ahnlichen Aufgaben standen dieses Produkt
nachgenutzt haben.
Aus dem Kreis der Nachnutzer hat sich die 'MILESS Community' entwickelt,
die aktiv Erfahrungen austauscht und zusammenarbeitet.
Mit der Nachnutzung haben sich auch viele Flexibilisierungsw"unsche
und zus"atzliche Anforderungen ergeben, insbesondere im Bereich
von digitalen Sammlungen und Archivl"osungen.
MILESS war urspr"unglich nur auf die lokalen Bed"urfnisse in Essen
zugeschnitten, so dass eine Erweiterung und Flexibilisierung 
des Metadatenmodells und die Anpassbarkeit des User Interfaces 
erforderlich wurden.

Aus diesen Anforderungen heraus entwickelte sich Anfang 2001 das
MyCoRe-Projekt \footnote{http://www.mycore.de}. 
Ziel von MyCoRe ist es, aufbauend auf den MILESS-Erfahrungen gemeinsam
ein neues System als Softwarebasis lokaler digitaler Bibliotheken zu
entwickeln.
Das Pr"afix 'My' steht dabei f"ur die Urspr"unge in der {\bf M}ILESS
Communit{\bf y}, das 'CoRe' f"ur Content Repository oder auch f"ur
Kern (core).
Die MyCoRe-Initiative ist f"ur alle Interessierten offen und hat sich
so organisiert, dass einzelne Mitglieder die Entwicklung bestimmter
Funktionsbereiche "ubernehmen, f"ur die sie die Analyse, das Design und
die Implementierung durchf"uhren.
Ein Architektur-Board von 4-5 Personen koordiniert dabei die gemeinsame
Entwicklung, legt Standards, Richtlinien und Schnittstellen fest, sammelt
die Anforderungen aller Mitglieder der Gemeinschaft und stellt die 
Integrationsf"ahigkeit der einzelnen Komponenten sicher.

Das vorliegende Dokument gibt eine grunds"atzliche Einf"uhrung in MyCoRe. 
Es richtet sich an alle, die sich einen "Uberblick "uber die Funktionalit"at und den
Leistungsumfang des MyCoRe-Softwaresystems verschaffen wollen.
Leser und Leserinnen werden dar"uber hinaus "uber die 
grunds"atzlichen Hard- und Softwarevoraussetzungen sowie "uber die
Nutzungs- und Lizenzbedingungen informiert. 
Auch werden sie erfahren, welche Funktionalit"aten f"ur MyCoRe zuk"unftig
geplant sind und in welchem Zeitrahmen die Entwicklungsschritte zu erwarten sind.
Eine Liste von Ansprechpartnern und Mitwirkenden am Projekt schlie"st diese
Dokumentation ab.

Eine ausf"uhrliche Dokumentation "uber die Installation des Softwaresystems sowie 
"uber die Installation und Konfiguration einer auf MyCoRe basierenden
Beispielanwendung mit einem Dublin Core Datenmodell findet sich in dem
'MyCoRe User Guide'.
Entwickler und Entwicklerinnen finden sehr detaillierte Informationen
zu der verwendeten Softwarearchitektur und zu den einzelnen Komponenten
im 'MyCoRe Internal Design Guide'.

Der besseren Lesbarkeit halber wird in diesem Text haupts"achlich die m"annliche 
Form verwendet (z.B. 'Benutzer' anstelle von 'Benutzer und Benutzerinnen'). 
Weibliche Personen sind aber selbstverst"andlich in gleicher Weise mit angesprochen.
 