Die folgende Aufstellung soll interessierte Nachnutzer des MyCore-Projektes
dar�ber informieren, welcher Funktionsumfang f�r die einzelnen Releases
von den Entwicklern vorgesehen ist. Da die Entwicklung des Projektes durch
das Developer-Team bedingt durch anderweitige Aufgaben  nicht immer i
kontinuierlich betrieben werden kann, ist eine Mitwirkung neuer Anwender
in h�chstem Ma�e erw�nscht.

\section{Funktionsumfang des Releases 0.8 - Juni 2004}
\label{sec:Planungen07}

{\bf Klassifikationen}
\begin{itemize}
\item Das Klassifikationsdatenmodell ist implementiert und funktioniert.
\item Klassifikationen werden in XML-Form gespeichert.
\item Kategorien k�nnen hierarchisch angeordenet werden, die Kategorien bekommen dann mehrstufige ID's.
\item Die Label der Kategorien k�nnen mehrsprachig sein.
\item Klassifikationen k�nnen per Commandline-Tools verwaltet werden.
\item Die f�r das MyCoRe-Beispiel ben�tigten Klassifikationen sind vorhanden.
\end{itemize}

{\bf Metadaten-Modell}
\begin{itemize}
\item Das allgemeine Datenmodell f�r die Metadaten ist implementiert.
\item Es sind 12 komplexe Grunddatentypen zur Gestaltung von Anwendungen implementiert.
\item Das Modell f�r die Datentypen gestattet eine mehrsprachige Speicherung.
\item Die Metadaten k�nnen per Commandline-Tools verwaltet werden.
\item Es gibt einen Migrationspfad von MILESS zu MyCoRe.
\item Unter Verwendung des IBM Content Managers 8.2 ist auch eine textindizierte Suche in den Metadaten m�glich. 
\item Es wurde ein Suchkonzept auf Basis eines vereinfachten und erweiterten XPath-Synatx implementiert.
\item Die Suche in den Daten kombiniert die Suche in den Metadaten mit einer Volltextsuche in den Dokumenten.
\item Das Design der Metadaten und der Suche gestattet eine einfache Erweiterung durch eigene Projekte der Anwender.
\end{itemize}

{\bf Internes File System}
\begin{itemize}
\item Es sind Funktionalit�ten zur Anbindung der eigentlichen Dokumente und multimedialen Objekte an die Metadaten implementiert.
\item Es ist eine zentrale Verwaltungseinheit zur Klassifizierung und Verteilung der einzelnen Objekte auf die Stores vorhanden.
\item Es sind folgende Stores implementiert: 
\begin{itemize}
\item FileSystem Store als Standardspeicher
\item Lucene Store zur Speicherung der Daten im FileSystem und Textindizierung von *.txt und *.pdf Dokumenten unter Lucene
\item IBM Content Manager 8.2 Store inklusive Volltextindizierung der Dokumente
\item Helix Server Store f�r multimediale Objekte
\item IBM VideoCharger Store f�r multimediale Objekte
\end{itemize}
\end{itemize}

{\bf Benutzer- und Rechtesystem}
\begin{itemize}
\item Die Benutzerdaten bestehen aus Nutzern welche in Gruppen angeordnet sind. Diese Gruppen k�nnen festgelegte Rechte haben.
\item Die Benutzerdaten k�nnen per Commandline-Tools verwaltet werden.
\item Das Benutzersystem hat eine API Schnittstelle (MCRUserManager) zu den
anderen Komponeneten von MyCoRe.
\item Via Servlet k�nnen die Privilegien einzelner Nutzer abgefragt werden.
\end{itemize}

{\bf Datenpr�sentation}
\begin{itemize}
\item Die Datenpr�sentation basiert auf XSLT 1.0.
\item In der Beispielanwendung wedern folgende M�glichkeiten der Darstellung gezeigt:
\begin{itemize}
\item einfache WEB-Dokumente und Seiten
\item Suche in der Datenbasis
\item Anzeige der Resultatsliste
\item Anzeige der Objekt-Metadaten
\item Anzeige der Attribute des Objektes und des Objektes selbst
\item Wechsel des Benutzers
\item Umschalten der Pr�sentation auf eine andere Sprache
\end{itemize}
\item Die Arbeit der implementierten Servlets l�uft Session-orientiert, d. h. alees Servlets kennen den konkerten Stand der einzelnen Sessions.
\end{itemize}

{\bf Remotezugriff und Services}
\begin{itemize}
\item Es wurde via HTTP eine M�glichkeit der Remote-Abfrage von Systemen geschaffen.
\item Es wurde ein OAI Client gem�� Standard 2.0 implementiert.
\end{itemize}

{\bf zus�tzliche Funktionalit�ten}
\begin{itemize}
\item Es wurde ein Tool zur Generierung von NBN's implementiert.
\end{itemize}

{\bf Dokumentation}
\begin{itemize}
\item Die Dokumentation gestattet die Installation des MyCoRe-Beispiels.
\item Mit ihrer Hilfe kann ein eigener Dokument-Server aufgebaut werden.
\item Sie zeigt erste Details der Gestaltung des Projektes.
\end{itemize}

\section{Funktionsumfang des Releases 0.9}
\label{sec:Planungen09}

{\bf Klassifikationen}
\begin{itemize}
\item Klassifikationen k�nnen mittels eines Editors erstellt werden.
\item Im MyCoRe-Beispiel sind weitere allgemeing�ltige Klassifikationen aus dem Bibliotheksumfeld abgelegt (z.B. PACS). Diese k�nnen f�r eigene Entwicklungen genutzt werden.
\end{itemize}


{\bf Metadaten-Modell}
\begin{itemize}
\item Die Struktur der bisherigen LegalEntities wird an die Personennormdatei angepasst.
\item Die Dokument-Metadaten werden werden um die Klassifikation Language (Dublin Core Feld 12) erweitert.
\end{itemize}

{\bf Internes File System}
\begin{itemize}
\item Optionale Komprimierung der Daten vor der Auslieferung.
\end{itemize}

{\bf Benutzer- und Rechtesystem}
\begin{itemize}
\item Die Benutzerdatenverwaltung wird �ber einen Editor gesteuert. 
\item Eine einfache Anbindung an LDAP-Server wird realisiert.
\item Es wird eine vereinfachte Form eines Access-Controll-Modells geben, welches den Zugriff auf die metadaten und Objekte regelt.
\end{itemize}

{\bf Datenpr�sentation}
\begin{itemize}
\item Es ist die Suche mittels Klassifikations-Browser im Ordner-Stil m�glich.
\item Implementation von Warenkorbfunktionen in das Beispiel.
\end{itemize}

{\bf Remotezugriff und Services}
\begin{itemize}
\item Weitere WebServices werden implementiert.
\end{itemize}

{\bf zus�tzliche Funktionalit�ten}
\begin{itemize}
\item Einbindung eines Digital Rights Management wie Wasserzeichen usw. in die Ausleiferungder Objekte
\end{itemize}

{\bf Dokumentation}
\begin{itemize}
\item Die Dokumentation ist vollst�ndig ausgearbeitet.
\end{itemize}

\section{Funktionsumfang des Releases 1.0}
\label{sec:Planungen10}

\begin{itemize}
\item Alle Kernfunktionalit�ten sind vollst�ndig vorhanden und getestet.
\item Die Beispielanwendung zeigt den Funktionsumfang voll auf.
\item Es liegen eine Reihe von Anwendungsprojekten zur Nachnutzung vor.
\item Die ausf�hrliche Dokumentation ist redaktionell �berarbeitet.
\end{itemize}

