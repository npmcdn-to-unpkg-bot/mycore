Das MyCoRe-Projekt verwendet f�r die Speicherung der Objekte (multimediale
Daten, Metadaten, Benutzerinformationen usw.) sowie f"ur die Suche in
Texten Software von Drittanbietern.
Dabei werden zwei grundlegend verschiedene Ans�tze unterst�tzt, und zwar
die Verwendung der kommerziellen Software wie des 'IBM Content Manager' der Firma IBM
sowie die Verwendung ausschliesslich frei verf�gbarer Software.
Jeder Ansatz hat seine eigenen Vor- und Nachteile, die an dieser Stelle
aber nur kurz angerissen werden sollen.
Bei der Entscheidung wird sicherlich die Ber"ucksichtigung des bereits 
bestehenden IT-Umfelds eine grosse Rolle spielen. 
In den folgenden Tabelle finden Sie die wesentlich
eingesetzten Softwarekomponenten entsprechend des gew�hlten Ansatzes.\\[2ex]

\small
\bottomcaption{MyCoRe Komponenten�bersicht f�r IBM CM}
\tablehead{\hline}
\tabletail{\hline}
\begin{supertabular}{|p{3cm}|p{11cm}|}
\hline
{\bf allg. Basis} & komerzieller IBM Content Manager \\ \hline
{\bf Metadaten Store} & IBM Content Manager 8.2 mit integrierter Text-Suchmashie
ne IBM NSE \\ \hline
{\bf Volltextsuche} & Diese ist f�r die Nutzung des IBM CM Ressource Managers vo
ll integriert. \\ \hline
{\bf Datenbank} & Es wird die mitgelieferte DB2 Datenbank benutzt. \\ \hline
{\bf Dokument Stores} & Filesystem \\
 & IBM CM Ressource Manager \\
 & IBM Video Carger \\
 & Helix Server \\ \hline
{\bf Web-Server} & IBM WebSphere \\ \hline
{\bf Servlet-Engine} & IBM WebSphere \\ \hline
{\bf OS} & IBM AIX \\
 & Sun Solaris \\
 & Linux \\
 & Microsoft Windows \\ \hline
{\bf Vorteile} & - Parametrische Metadaten-Werte k�nnen ggf. auch volltextindize
irt und so gesucht werden werden. \\
 & - Viele Dokumenttypen lassen sich automatisch volltextindizieren. \\
 & - Das Basisprodukt kommt von einem Hersteller. \\ \hline
{\bf Nachteile} & - Sehr komplexe Installation des ben�tigten Content Manager /
WebSphere Systems. \\
 & - Abh�ngigkeit von den IBM Lizenz-Bedingungen. \\
\hline
\end{supertabular}
\normalsize
\\[2ex]
\small
\bottomcaption{MyCoRe Komponenten�bersicht f�r freie L�sung}
\tablehead{\hline}
\tabletail{\hline}
\begin{supertabular}{|p{3cm}|p{11cm}|}
\hline
{\bf allg. Basis} & L�sung auf freien Komponenten \\ \hline
{\bf Metadaten Store} & Es wird die freie XML:DB eXist benutzt. \\ \hline
{\bf Volltextsuche} & Es wird die freie Text-Suchmaschine Lucene benutzt. \\ \hl
ine
{\bf Datenbank} & Es wird die freie Datenbank MySQL benutzt. \\ \hline
{\bf Dokument Stores} & Filesystem \\
 & Lucene f�r Textdokumente \\
 & IBM Video Carger \\
 & Helix Server \\ \hline
{\bf Web-Server} & Apache / Tomcat \\ \hline
{\bf Servlet-Engine} & Tomcat \\ \hline
{\bf OS} & Linux \\
 & Microsoft Windows \\ \hline
{\bf Vorteile} & - Geringe Kosten durch die Nutzung freier Komponeneten. \\
 & - Keine direkte Bindung an einen Hersteller. \\
 & - Die Komponenten sind gut getestet und dokumentiert. \\ \hline
{\bf Nachteile} & - Keine L�sung der Basis aus einer Hand. \\
 & - Viele unterschiedliche Quellen der Komponenten. \\
 & - Schwierige Installation auf nicht-Linux/MS Systemen. \\
\hline
\end{supertabular}
\normalsize

Die Hardwareanforderungen h"angen nat"urlich auch von der erforderlichen Performance,
der Ausfallsicherheit usw. ab. 
Ein aktuell (Februar 2004) verf"ugbarer PC mit 2 GB Hauptspeicher oder
mehr ist aber in der Lage, ein performantes System auf Basis der frei
verf"ugbaren Software zu realisieren.
Soll der IBM Content Manager zum Einsatz kommen, so k"onnen die minimalen
Hardwareanforderungen f"ur die einzelnen Komponenten (Video Charger, Content
Manager, DB2) auf den Webseiten der IBM eingesehen werden.
Prinzipiell kann man aber sagen, dass durch den Einsatz des IBM WebSphere
Application Server h"ohere Anforderungen an den Hauptspeicher gestellt
werden. 
Der f"ur die Content Manager Prozesse verf"ugbare Hauptspeicher 
sollte nicht unter 2 GB liegen.\\[2ex]

