Das MyCoRe-Projekt verwendet f"ur die Speicherung der Objekte (multimediale
Daten, Metadaten, Benutzerinformationen usw.) sowie f"ur die Suche in
Texten Software von Dritten.
Dabei werden zwei grundlegend verschiedene Ans"atze unterst"utzt, und zwar
die Verwendung der kommerziellen Software 'Content Manager' der Firma IBM
sowie die Verwendung ausschliesslich frei verf"ugbarer Software.
Jeder Ansatz hat seine eigenen Vor- und Nachteile, die an dieser Stelle
aber nicht diskutiert werden sollen.
Vielmehr wollen wir es dem Anwender "uberlassen, in welchem System er 
f"ur seine Anwendung die gr"o"sten Vorteile sieht. 
Bei der Entscheidung wird sicherlich die Ber"ucksichtigung des bereits 
bestehenden IT-Umfelds eine grosse Rolle spielen. 
Nachfolgend finden Sie eine Tabelle der wesentlichen eingesetzten Softwarekomponenten 
entsprechend des gew�hlten Ansatzes.
\small
\bottomcaption{"Ubersicht der MyCoRe Softwarekomponenten}
\tablehead{\hline}
\tabletail{\hline}
\begin{supertabular}{|p{2cm}|p{6cm}|p{6cm}|}
\hline
 & {\bf IBM CM Basis} & {\bf freie Software} \\[1,5ex] \hline
Metadaten Store 
 & IBM CM 8.2 - parametrische und Volltextsuch mittels XPath Abfragen 
 & eXist - parametrische Suche mittels XPath Abfragen \\ \hline
TextSearch 
 & IBM DB2 NSE 
 & lucene  \\ \hline
Datenbank 
 & IBM DB2 8.1 oder Oracle 
 & MySQL 4.x \\ \hline
Objekt 
 & Filesystem, 
 & Filesystem,  \\
Store 
 & IBM CM 8.2 Ressource Manager, 
 &  \\ 
 & IBM Video Charger 8, 
 & IBM Video Charger 8, \\
 & Helix Server 
 & Helix Server  \\  \hline
Systembasis 
 & IBM AIX 
 & alle Systeme mit Java \\
 & Sun Solaris & \\
 & Linux & \\
 & MS Windows (Server) & \\
\hline
\end{supertabular}
\normalsize

Die Hardwareanforderungen h"angen nat"urlich auch von der erforderlichen Performance,
der Ausfallsicherheit usw. ab. 
Ein aktuell (November 2003) verf"ugbarer PC mit 1 GB Hauptspeicher oder
mehr ist aber in der Lage, ein performantes System auf Basis der frei
verf"ugbaren Software zu realisieren.
Soll der IBM Content Manager zum Einsatz kommen, so k"onnen die minimalen
Hardwareanforderungen f"ur die einzelnen Komponenten (Video Charger, Content
Manager, DB2) auf den Webseiten der IBM eingesehen werden.
Prinzipiell kann man aber sagen, dass durch den Einsatz des IBM WebSphere
Application Server h"ohere Anforderungen an den Hauptspeicher gestellt
werden. 
Der f"ur die Content Manager Prozesse verf"ugbare Hauptspeicher 
sollte nicht unter 2 GB liegen.
