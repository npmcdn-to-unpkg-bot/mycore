%
%
\subsection{Das User- und Rechtesystem}
%
%
Im MyCoRe-Sample wird zur Demonstartion des User- und Rechtesystems auch eine konkrete Beispielkonfiguration f�r diesen Bereich mitgeliefert. Ein allgemeiner �berblick zu diesem Teilsystem wurde bereits weiter oben gegeben.\\[2ex]
Im Konfigurationsverzeichnis des Samples unter {\it \$MYCORE\_SAMPLE\_HOME/config} ist zuerst das entsprechende Property-File mit dem Namen {\it mycore.properties.user} n�her zu betrachten. Sie finden hier Festlegungen in der folgenden Form:
\begin{verbatim}
# The name of the Java class of the UserStore
MCR.userstore_class_name=org.mycore.backend.sql.MCRSQLUserStore

# SQL tables for the user management
MCR.users_store_sql_table_users=MCRUSERS
MCR.users_store_sql_table_groups=MCRGROUPS
MCR.users_store_sql_table_group_members=MCRGROUPMEMBERS
MCR.users_store_sql_table_group_admins=MCRGROUPADMINS
MCR.users_store_sql_table_privileges=MCRPRIVS
MCR.users_store_sql_table_privs_lookup=MCRPRIVSLOOKUP

# The configuration for the superuser
MCR.users_superuser_username=gandalf
MCR.users_superuser_userpasswd=alleswirdgut
MCR.users_superuser_groupname=zauberer

# The configuration for the guestuser
MCR.users_guestuser_username=aragorn
MCR.users_guestuser_userpasswd=mensch
MCR.users_guestuser_groupname=menschen
\end{verbatim}
Der Eintag des Parameters {\bf MCR.userstore\_class\_name} beschreibt den zu verwendenden Persitence Layer f�r die vom User-System verwendeten relationalen Datenbanken. Wir haben hier einen generischen Store entwickelt der mit allen SQL-konformen Datenbanken wie DB2, MySQL usw. laufen sollte. Sie m�ssen also diesen Wert normalerweise nicht �ndern.\\[2ex]
Die n�chste Sektion in diesem Property-File legt die Tabellennamen der f�r das User-System ben�tigten Tabellen fest. F�r die Ausf�hrung des Beispieles k�nnen Sie die Einstellung �bernehmen. Achten Sie jedoch sp�ter darauf, dass Sie die Tabellennamen nur in einem Projekt verwenden!\\[2ex]
Die letzten beiden Abschnitte legen Vorgaben f�r zwei Benutzer fest, welche automatisch bei der Initialisierung des User-Systems angelegt werden. Der erste ist der Superuser innerhalb dieses MyCoRe-Projektes, der andere ein ganz simpler Anwender. Auch diese Werte k�nnen Sie f�r den Anfang so lassen, f�r ein eigenes Projekt, wie es weiter unten beschrieben wird, sollten Sie hier Anpassungen vornehmen. Eine Liste aller Gruppen und Benutzer des Beispieles finden Sie im Abschnitt zur allgemeinen Beschreibung dieses Subsystems.\\[2ex]
Alle f�r das Beispiel erforderlichen Privileg-, Gruppen- und Benutzerdateien stehen im Verzeichnis {\it \$MYCORE\_SAMPLE\_HOME/config/users}. Sie k�nnen diese mit {\tt cd \$MYCORE\_SAMPLE\_HOME; bin/build.sh userdb} oder {\tt cd \$MYCORE\_SAMPLE\_HOME; ant userdb} in einem Arbeitsgang laden.\footnote{Welches Kommando Sie nutzen h�ngt von Ihrer Umgebungseistellung ab.} Der Aufruf initialisiert zuerst das User-System und legt die in der Konfiguration angegebenen Tabellen an. Anschlie�end werden die Privilegien, Gruppen und Benutzer geladen. Bitte achten Sie darauf, dass die vorgegebene Reihenfolge eingehalten wird, da es innerhalb des Beispiels Abh�ngigkeiten zwischen Usern und Gruppen gibt.\\[2ex]
Nun sollten Sie �ber ein komplettes User-System, wie es zur Demonstartion des Samples erforderlich ist, verf�gen.

