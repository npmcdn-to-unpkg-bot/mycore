\documentclass[a4paper,12pt]{book}
\usepackage{german}
\usepackage{fancyheadings} 
\usepackage[dvips]{graphicx}
\DeclareGraphicsRule{.jpg}{eps}{jpeg2ps #1}
\pagestyle{fancyplain} 
\addtolength{\headwidth}{\marginparsep} 
\addtolength{\headwidth}{\marginparwidth} 
\renewcommand{\chaptermark}[1]% 
      {\markboth{#1}{}} 
\renewcommand{\sectionmark}[1]% 
      {\markright{\thesection\ #1}} 
\lhead[\fancyplain{}{\bfseries\thepage}]% 
      {\fancyplain{}{\bfseries\rightmark}} 
\rhead[\fancyplain{}{\bfseries\leftmark}]% 
      {\fancyplain{}{\bfseries\thepage}} 
\cfoot{} 
\begin{document}
%-------   Vorspann
\title{MyCoRe Starting Guide}
\author{
    Frank L"utzenkirchen\\
    Jens Kupferschmidt}
\maketitle
\setcounter{secnumdepth}{10}
\chapter*{Vorwort}
Das Dokument soll eine grunds"atzliche Einf"uhrung in MyCoRe geben.

\tableofcontents
\listoffigures
\listoftables
%-------    Hauptteil
\chapter{Leistungsumfang von MyCoRe}
\includegraphics{bechtle.jpg}
\chapter{Grunds"atzliche Hard- und Software-Voraussetzungen}
\chapter{Nutzungs- und Lizenzbedingungen}
\chapter{Unterst"utzte Features der Releases / Roadmap}
\chapter{Grunds"atzliches Modell (MCRObjectID, Vererbung, XML, Stylesheets, usw.)}
\chapter{Mitwirkende am Projekt / Ansprechpartner}
\end{document}
