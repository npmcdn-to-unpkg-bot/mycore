\chapter{Tabellen}
Tabellen stehen in der Umgebung tabular und ihrer um weiter Funktionalt�t erweiterten Abk�mmlinge (supertabular, tabularx). Im Allgemeinen wird auch die Tabelle als flie�endes Objekt realisiert und die gesamte Tabelle in die Umgebung table gesetzt. Als Parameter werden der Tabelle die Ausrichtung der Eintr�ge �bergeben, also zentriert (c), linksb�ndig (l) oder rechtsb�ndig (r). Die Spaltenbreite wird automatisch ermittelt. Jede Zeile wird mit \textbackslash\textbackslash abgeschlossen, jede Spalte mit \&. Die letzte Zeile wird mit dem letzten Eintrag beendet. Horizontale Linien werden mit \verb+\hline+ eingezogen, vertikale Linien im Ausrichtungsblock (\textbar). Aus typographischen Gesichtspunkten empfehle ich jedoch keine vertikalen Linien und nur folgende horizontalen (\begin{verbatim} \toprule, \midrule, \bottomrule \end{verbatim})  zu verwenden. Um die Lesbarkeit noch weiter zu verbessern kann man am Ende jeder Zeile mit \verb+ \addlinespace+ einen vertikalen Abstand einf�gen zwichen den Zeilen einf�gen.
Ich empfehle folgende Tabellenumgebungen zu verwenden \verb+tabular, tabularx, supertab+. 
Man kann zwar alle m�glichen Tabellen unter \LaTeX erstellen, jedoch ist die in aller Regel nicht ganz so trivial. F�r Windows empfehle ich LaTable\footnote{Download: \url{http://www.g32.org/files/latable/latable-0_7_2.zip}} oder excel2latex\footnote{Download: \url{http://www.jam-software.com/freeware/xl2latex.zip} \& Readme: \url{http://www.jam-software.com/freeware/excel2latex_readme_en.txt}} um komplexte Tabellen zu erzeugen. Unter Linux leistet kile oder gnumeric ganz gute Dienste. Verwendet man diese Tools kann ich mir eigentlich jede weitere Erkl�rung sparen. Doc nur noch ganz kurz ein paar Worte zu den einzelnen Umgebungen

\section{tabular}
Tabular ist die einfachste Tabellenform. Verwendet werden sollte sie bei einfachen Tabellen, die sich nicht �ber mehrere Seiten (siehe Abschnitt \ref{supertab}) erstrecken und nicht �ber die Seitenbreite hinausragt. da dies im allgemeinen schwierig ist vorrauszusehen empfielt sich die tabularx Umgebung. Prinzipiell

\section{tabularx}

Mit diesem Package kann man Spalten von Tabellen gleich gro� (=breit) machen. Die Gr��e bestimmt sich durch die Gesamtbreite der Tabelle. Tabularx Spalten werden mit einem X in die Spaltendefinition aufgenommen. Wichtig ist noch (wie bei Tabular*), dass eine Tabellenbreite nach dem \verb+\begin{tabularx}+ Befehl angegeben wird, damit Latex wei� wie breit die Spalten zu machen sind. Spalten, die mit X eingef�gt werden sind linksb�ndig. Es ist m�glich mehrere Tabularx Spalten im Header zu definieren, falls man beispielsweise rechtsb�ndige oder centrierte Spalten m�chte (Befehle, die in den Header geh�ren sind unter dem Beispiel angef�hrt). Diese unterschiedlichen Tabularx Spaltentypen k�nnen innerhalb eine Tabularx Tabelle gleichzeitig verwendet werden und werden unabh�ngig von der Textausrichtung gleich breit formatiert.

Beispiel f�r eine Tabularx Tabelle:

\begin{verbatim}
\newcolumntype{Y}{>{\raggedleft\arraybackslash}X}
%neues Tabellenspaltenformat um TABULARX Spalten zu zentrieren
\newcolumntype{Z}{>{\centering\arraybackslash}X}

\begin{table}[htb] 
\caption{Tabellen�berschrift} %�berschrift der Tabelle 
\label{label} %Label der Tabelle 
%
\begin{tabularx}{\textwidth}{rlXYZp{1cm}}
\toprule
rechts    &  links  &    X &        Y &        Z    &     p \\ 
\midrule
1         &  2      & LINKS &  RECHTS &       MITTE &  definiert \\ 
\bottomrule
   
\end{tabularx} 

\end{table}
\end{verbatim}


\newcolumntype{Y}{>{\raggedleft\arraybackslash}X}
%neues Tabellenspaltenformat um TABULARX Spalten zu zentrieren
\newcolumntype{Z}{>{\centering\arraybackslash}X}

\begin{table}[htb] 
\caption{Tabellen�berschrift} %�berschrift der Tabelle 
\label{label} %Label der Tabelle 
%
\begin{tabularx}{\textwidth}{rlXYZp{1cm}}
\toprule
       rechts    &  links         &      X &       Y &        Z &         p \\ \midrule
       1         &         2      & LINKS &       RECHTS &       MITTE &       definiert \\ 
       \bottomrule
   
\end{tabularx} 

\end{table}

Spalten die gleich gro� werden sollen und rechtsb�ndig sind anstatt als X als Y einzuf�gen, zentrierte Spalten als Z (oder nach eigener Definition). In die oben angef�hrten Spaltendefinitionen k�nnen auch Schriftgr��enbefehle eingef�gt werden und/oder Textformatierungen. Wichtig ist, dass den Befehlen der \verb+\arraybackslash+ folgt.



\section{Mehrseitige Tabellen - Die supertabular Umgebung}\label{supertab}
Supertabular wird f�r mehrseitige Tabellen verwendet. Mit mpsupertabular ist es zudem m�glich Fu�noten in den Tabellen zu setzen. F�r weitere Optionen ist ein Blick in die Dokumentation (liegt jeder \LaTeX Distribution bei) supertabular.dvi zu werfen. Da ist eigentlich alles sehr gut erkl�rt. Prinzipiell kann aber einfach statt der tabular die supertabular Umgebung verwendet werden. 
 
\begin{verbatim}
 \begin{supertabular}
    ...
  \end{supertabular} 
\end{verbatim} 



\section{Eine lange Tabelle querformatig einf�gen}
Bettet man eine tabular, tabularx Umgebung mit 
\begin{verbatim}
\begin{sidewaystable}
  \begin{tabular}
    ...
  \end{tabular} 
\end{sidewaystable}
\end{verbatim} 

ein. so wird die Tabelle im Querformat auf die Seite gedruckt.


\section{Bunte Tabellen}
Ist eigentlich selbsterkl�rend.
\begin{verbatim}
\begin{tabular}{%
    >{\columncolor[gray]{0.9}}p{0.3\linewidth}  % graue Spalte
    p{0.3\linewidth}                            % normale Spalte
    >{\columncolor[gray]{0.9}}p{0.3\linewidth}  % graue Spalte
    }\toprule
  Kon\-ti\-nent & Berg & H"ohe [m]\\ \midrule
  Asien & Chomolungma & 8848\\
  S"udamerika ist ein Land das ich sehr sch"on finde& Aconcagua & 6962\\
  Nord\-ameri\-ka & Denali & 6194\\
  Afrika & Kilimandscharo & 5895\\
  Antarktis & Mount Vinson & 4897\\
  Australien & Carstensz Pyramid & 4884\\
  Europa & Elbrus & 5642\\
  \bottomrule
\end{tabular}
\end{verbatim}

\begin{tabular}{%
    >{\columncolor[gray]{0.9}}p{0.3\linewidth}  % graue Spalte
    p{0.3\linewidth}                            % normale Spalte
    >{\columncolor[gray]{0.9}}p{0.3\linewidth}  % graue Spalte
    }\toprule
  Kon\-ti\-nent & Berg & H"ohe [m]\\ \midrule
  Asien & Chomolungma & 8848\\
  S"udamerika ist ein Land das ich sehr sch"on finde& Aconcagua & 6962\\
  Nord\-ameri\-ka & Denali & 6194\\
  Afrika & Kilimandscharo & 5895\\
  Antarktis & Mount Vinson & 4897\\
  Australien & Carstensz Pyramid & 4884\\
  Europa & Elbrus & 5642\\
  \bottomrule
\end{tabular}