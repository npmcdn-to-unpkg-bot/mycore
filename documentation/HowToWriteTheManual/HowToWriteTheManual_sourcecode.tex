\chapter{Sourcecode Formatierungen}
Das einf�gen von Sourcecode kann entweder �ber die Umgebung verbatim erfolgen oder was sich speziell bei Programmiersprachen anbietet ist das Paket listings.

\section{Die \textit{verbatim} Umgebung}

\begin{verbatim}
hier steht der Sourcecode und alle Zeichen werden
nicht von Latex interpretiert und so wie sie darstehen
auch dargestellt
\end{verbatim}

\section{Die \textit{lstlisting} Umgebung}
Eine umfassende Dokumentation ist zu finden im listings.pdf, dass der Distribution bei liegt oder unter \url{http://www.ctan.org/tex-archive/macros/latex/contrib/listings/listings.pdf} heruntergeladen werden kann. Falls das Paket nicht in ihrer Distribution enthalten ist kann es einfach installiert werden. Siehe dazu \ref{sec:installation_pakete}
F�r die MyCoRe Dokumentation sind die folgenden Beispiele im Prinzip gen�gend.
Danach k�nnen Sie Ihr Code-Listing wie folgt einbinden:

\begin{verbatim}
\lstset{language=XML,fancyvrb=true,frame=btlr}
\lstset{breaklines,prebreak={\space\McrHookSign}}
\begin{lstlisting}[caption={Eintrag_Listingsverz]XML Beispiel},%
		label=lst:XmlBeispiel]{frame=single}
<structure>
 <parents class="MCRMetaLinkID">
  <parent xlink:type="locator" xlink:href="...mcr_id..." />
 </parents>
...
</structure>
\end{lstlisting}
\end{verbatim}

\lstset{language=XML,fancyvrb=true,frame=btlr,breaklines,prebreak={\space\McrHookSign}}
\begin{lstlisting}[caption={[Eintrag_Listingsverz]XML Beispiel},label=lst:XmlBeispiel]{frame=single}
<structure>
 <parents class="MCRMetaLinkID">
  <parent xlink:type="locator" xlink:href="...mcr_id..." />
 </parents>
 <children class="MCRMetaLinkID">
  <child xlink:type="locator" xlink:href="...mcr_id..." xlink:label="..." xlink:title="..." />
  ...
 </children>
 <derobjects class="MCRMetaLinkID">
  <derobject xlink:type="locator" xlink:href="...mcr_id..." xlink:label="..." xlink:title="..." />
  ...
 </derobjects>
</structure>
\end{lstlisting}

M�chten Sie den Code aus einer externen Datei einf�gen, so verfahren Sie wie folgt:

\begin{verbatim}
\lstset{language=Java, basicstyle=\small, commentstyle=\color{grey}}
\lstset{linewidth=\textwidth, showstringspaces=false}
\lstset{numbers=left, stepnumber=5, numbersep=10pt}
\lstset{fancyvrb=true,frame=btlr,breaklines,prebreak={\space\McrHookSign}}
\lstinputlisting[caption=Multi-Page Java Code,label=lst:java]{example.java}
\end{verbatim}

\lstset{language=Java, basicstyle=\small, commentstyle=\color{grey}}
\lstset{linewidth=\textwidth, showstringspaces=false}
\lstset{numbers=left, stepnumber=5, numbersep=10pt}
\lstset{fancyvrb=true,frame=btlr,breaklines,prebreak={\space\McrHookSign}}
\lstset{rulesepcolor=\color{black}}
\lstinputlisting[caption=Multi-Page Java Code,label=lst:java]{example.java}



