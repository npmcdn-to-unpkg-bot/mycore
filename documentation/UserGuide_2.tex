%
%
\chapter{Download und Installation des MyCoRe Kerns}
%
%
\section{Download des MyCoRe Kerns}
%
%
Das MyCoRe Projekt wird f"ur alle unterst"utzten Systeme "uber das CVS Repository
ausgeliefert. Das Holen der aktuellen Version erfolgt mit dem Kommando 
\begin{center}
{\tt cvs -d :pserver:anoncvs@server.mycore.de:/cvs checkout mycore} 
\end{center}
Nach dem erfolgreichen Checkout erhalten Sie folgende Dateistruktur:\\[2ex]
\bottomcaption{Dateistruktur des MyCoRe Kernes}
\tablehead{\hline}
\tabletail{\hline}
\begin{supertabular}{|p{5cm}|p{10cm}|}
\hline
{\bf mycore} &  Das Root-Verzeichnis des MyCoRe-Kerns \\
\quad {\bf bin} & Das Verzeichnis der Shellscripte \\
\qquad build.sh & Shellscript zum Compilieren unter einem UNIX-System \\
\qquad build.cmd & Shellscript zum Compilieren unter einem UNIX-System \\
\qquad mycore.sh & Shellscript zur Arbeit mit dem Commandline-Interface unter UNIX \\
\qquad mycore.cmd & Shellscript zur Arbeit mit dem Commandline-Interface unter Windows \\
\qquad setup\_cm7.sh & Shellscript f"ur eine UNIX-Umgebung mit IBM Content Manager 7 \\
\qquad setup\_cm7.cmd & Kommandoscript f"ur eine Windows-Umgebung mit IBM Content Manager 7 \\
\qquad setup\_cm8.sh & Shellscript f"ur eine UNIX-Umgebung mit IBM Content Manager 8.1/8.2 \\
\qquad setup\_cm8.cmd & Kommandoscript f"ur eine Windows-Umgebung mit IBM Content Manager 8.1/8.2 \\
\qquad setup\_xindice.sh & Shellscript f"ur eine UNIX-Umgebung mit Xindice \\
\qquad setup\_xindice.cmd & Kommandoscript f"ur eine Windows-Umgebung mit Xindice \\
\quad {\bf documentation} & Dokumentationen zu MyCoRe \\
\quad {\bf lib} & Noterndige zus"atzliche Java-Bibliotheken \\
\quad {\bf schema} & XMLSchema Dateien, die anwendungsunabh"angig sind \\
\quad {\bf sources/org/mycore} & Die Wurzel des MyCoRe-Source-Baumes \\
\qquad {\bf datamodel} & Klassen zum Datenmodell \\
\quad \qquad {\bf classifications} & Klassen zur Arbeit mit den Klassifikationen \\
\quad \qquad {\bf ifs} & Klassen zur Arbeit mit dem Internal File System \\
\quad \qquad {\bf metadata} & Klassen zur Arbeit mit den Metdaten \\
\qquad {\bf common} & Klassen, die im gesamten Projekt ben�tigt werden \\
\quad \qquad {\bf xml} & Allgemeine Klassen zur XML Verarbeitung \\
\qquad {\bf backend} & Klassen f"ur die verschiedenen Data Stores \\
\quad \qquad {\bf cm7} & Klassen zur Nutzung des IBM Content Manager 7 \footnote{Die Klassen f�r den IBM Content Manager 7 werden nicht mehr weiterentwickelt!} \\
\quad \qquad {\bf cm8} & Klassen zur Nutzung des IBM Content Manager 8.2 \\
\quad \qquad {\bf filesystem} & Klassen zum Content Store im lokalen Filesystem \\
\quad \qquad {\bf realhelix} & Klassen zum Content Store in einem Helix-Server \\
\quad \qquad {\bf remote} & Klassen zum Zugriff auf Remote-MyCoRe-Daten \\
\quad \qquad {\bf sql} & Klassen zum Zugriff auf relationale Datenbanken mittels SQL-Standart \\
\quad \qquad {\bf videocharger} & Klassen zum Content Store in einen IBM Videocharger \\
\qquad {\bf frontend} & Klassen f"ur die Frontends des MyCoRe-Systems \\
\quad \qquad {\bf cli} & Klassen des Commandline-Tools \\
\quad \qquad {\bf editor} & Klassen zug Gestaltung von Editoren \\
\quad \qquad {\bf servlets} & Klassen zu Gestaltung von Servlets \\
\qquad {\bf services} & Klassen f"ur weiterf"uhrende Services des MyCoRe-Projektes \\
\quad \qquad {\bf nbn} & Klassen zur Arbeit mit NBN's \\
\quad \qquad {\bf oai} & Klassen zur Arbeit mit OAI Komponenten \\
\quad \qquad {\bf query} & Klassen zur Arbeit mit dem MyCoRe internen Query-System \\
\qquad {\bf user} & Klassen des User- und Rechteverwaltungssystems \\
\quad build.xml & Konfigurations-File f�r die Arbeit mit ANT \\
\quad license.txt & Das Lizenz-File des MyCoRe-Projektes, bitte lesen Sie dieses File aufmerksam durch, bevor Sie MyCore einsetzen. \\
\hline
\end{supertabular}
%
%
\section{Konfiguration und "Ubersetzten des Kerns}
%
%
Nach dem checkout der Files muss als erstes dieses Paket "ubersetzt werden. Die Arbeit erfolgt in den unten angegebenen Schritten.\\
\begin{enumerate} 
\item Zur Anpassung der erforderlichen Umgebung gibt es zwei Wege. Entweder alle notwendigen Variablen wie CLASSPATH usw. werden in den entsprechenden Nutzer-Profile gesetzt oder ({\bf empfohlen}) sie setzen an dieser Stelle nur die Environment-Variable {\bf MYCORE\_HOME} in Ihrer Systemumgebung und kopieren das entsprechende Template {\it setup\_....sh} (bzw. {\it setup\_....cmd}) im Verzeichnis {\it bin} dort nach {\it setup.sh} (bzw. {\it setup.cmd} f"ur die Windows Plattform) F"ur die unterst"utzten Systeme stehen jeweils entsprechende Files bereit. Passen Sie nun das Setup-Script an die eigene Konfiguration hinsichtlich der ben"otigten Software (Pfade zu ANT, Java usw.) an. 
\item Als n"achstes ist im File {\it \$MYCORE\_HOME/build.xml} die Zeile \\
\begin{center}
{\tt \begin{verbatim}<property name="persistency" value="cm7" />\end{verbatim}}
\end{center}
entsprechend des gew"ahlten Persistence-Layers anzupassen. M"ogliche Werte sind {\it cm7}, {\it cm8} und {\it xmldb}.
\item Starten Sie nun die "Ubersetzung mit einem Aufruf von {\it ant} oder ({\bf empfohlen}) der Ausf"uhrung des Shell-Scripts {\it bin/build.sh} (bzw. {\it bin/build.cmd}). Bei der Abarbeitung des Kommandos ohne Parameter wird Ihnen eine Auswahl an m"oglichen Parametern angeboten. So k"onnen Sie hier zum Beispiel alle Klassen kompilieren oder nur die Javadoc-Dokumentation bzw. diese Dokumentation erstellen. 
\item Um die f"ur die MyCoRe-Anwendung ben"otigten Java-Klassen zu erzeugen starten Sie einfach \\
\begin{center}
{\it \$MYCORE\_HOME/bin/build.sh jar }
\end{center} 
\end{enumerate} 
Abh"angig von Ihrer Wahl beim build werden weitere Unterverzeichnisse im Verzeichnisbaum unter {\it \$MYCORE\_HOME} erstellt: 
\begin{itemize} 
\item mycore/classes $\longrightarrow$ enth"alt die Java Klassen 
\item mycore/javsadocs $\longrightarrow$  enth"alt die Java Dokumentation 
\end{itemize} 
%
%
