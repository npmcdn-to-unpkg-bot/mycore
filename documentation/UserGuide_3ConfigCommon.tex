%
%
\subsection{Grundlegende Konfigurationen}
%
%
Dieser Abschnitt besch�ftigt sich mit der Konfiguration der Beispielanwendung in allgemeinen Bereichen wie JDBC, Logger, usw. . Alle Konfigurationen f�r das Mycore-Beispiel stehen unter {\it \$MYCORE\_SAMPLE\_HOME/config}.
%
%
\subsubsection{Allgemeines}
%
%
In der Konfigurationsdatei {\it mycore.properties.privat} legen Sie im Parameter {\bf MCR.persistence\_type} fest, welchen Persitence Layer sie nutzen wollen. M�gliche Werte sind {\bf cm7}, {\bf cm8} oder {\bf xmldb}.
%
%
\subsubsection{JDBC-Treiber}
%
%
Im MyCoRe-Projekt werden ein Teil der Organisations- und Metadaten in klassischen relationalen Datenbanken gespeichert. Um die Arbeit mit verschiedenen Anbietern m�glichst einfach zu gestalten, wurde die Arbeit mit dieser Datenbank gegen die JDBC-Schnittstellen programmiert.\\
In der Konfigurationsdatei {\it mycore.properties.privat} legen Sie im Parameter{\bf MCR.persistence\_sql\_database\_url} fest, welcher JDBC-Treiber genommen werden soll. Weiterhin k�nnen Sie die minimale und maximale Anzahl der Connections festlegen.
\begin{verbatim}
# JDBC parameters for connecting to DB2
#MCR.persistence_sql_database_url=jdbc:db2:LIB
#MCR.persistence_sql_driver=COM.ibm.db2.jdbc.app.DB2Driver

# JDBC parameters for connecting to MySQL
MCR.persistence_sql_database_url=jdbc:mysql://localhost/mycore?user=ODBC
MCR.persistence_sql_driver=org.gjt.mm.mysql.Driver

MCR.persistence_sql_init_connections=1
MCR.persistence_sql_max_connections=5
\end{verbatim} 


